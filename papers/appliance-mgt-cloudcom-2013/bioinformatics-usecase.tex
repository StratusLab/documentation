\section{Bioinformatics Appliance Metadata}
\label{sec:bioinfo}

In addition to the reference cloud infrastructure, a cloud
infrastructure devoted to biology (IDB cloud~\cite{idbcloud}), with a
separate Marketplace instance, has been deployed at
IBCP\footnote{Institut de Biologie et Chimie des Prot\'eines in Lyon,
  France}.

Biologists and bioinformaticians frequently combine multiple software
packages (from the thousands available in the international community)
to study their data with their own or public analysis pipelines. With
the advent of the cloud, experts now create customized appliances, but
these are often not adequately described or easily located.  Helping
scientists easily identify suitable appliances containing the required
software packages is essential.

Operating a separate Marketplace with a limited thematic scope already
allows users to find appropriate appliances more easily.  It also
reduces ``noice'' in the central Marketplace from iterative attempts
to create working appliances.  More importantly, however, it allows
visibility contraints for the bioinformatic appliances, such as
confidentiality for specific projects, to be respected.

To further facilitate searches for appropriate appliances,we have
extended the generic appliance metadata schema
(Fig.~\ref{fig:metadata-example}) with additional elements related to
bioinformatics tools (Fig.~\ref{fig:biotool-schema})\@. These metadata
can be used to select suitable bioinformatics appliances containing
the required tools, searching for the tools themselves (e.g. BLAST or
ClustalW2) or the type of analysis (e.g. sequence similarity or
multiple sequence alignment).  Appliance creators enhance the
descriptions by appending `bio:tool' entries in the appliance metadata
(see Fig.~\ref{fig:biotool-example}).

As with all metadata, the biotool information is indexed by the
Marketplace, allowing bioinformaticians and biologists to search the
Marketplace with a SPARQL query to find an appropriate appliance.
SPARQL, however, operates at a rather low level, so to further
simplify searches, we linked the Marketplace with the IDB
bioinformatics web portal. On the portal we synchronize (manually at
this point) the list of suitable appliances from the Marketplace. The
portal provides users with popup menus to filter appliances according
to the above criteria.  With the Marketplace, additional metadata, and
the portal it is trivial for users to find and run the right
appliance.

\begin{figure}
\scriptsize
 \begin{center}
    \begin{verbatim}
<bio:tool rdf:parseType="Resource">
  <dcterms:identifier>tool-name</dcterms:identifier>
  <dcterms:description>
    tool description
  </dcterms:description>
  <slterms:version>tool.version</slterms:version>
</bio:tool>
    \end{verbatim}
  \end{center}
 \caption{Schema of the `bio:tool' metadata}
 \label{fig:biotool-schema}
\end{figure}


\begin{figure}
\scriptsize
 \begin{center}
    \begin{verbatim}
<bio:tool rdf:parseType="Resource">
  <dcterms:identifier>blast+</dcterms:identifier>
  <dcterms:description>
    BLAST: sequence similarity search
  </dcterms:description>
  <slterms:version>2.2.27</slterms:version>
</bio:tool>
<bio:tool rdf:parseType="Resource">
  <dcterms:identifier>clustalw2</dcterms:identifier>
  <dcterms:description>
    ClustalW2: multiple sequence alignment
  </dcterms:description>
  <slterms:version>2.1</slterms:version>
</bio:tool>
    \end{verbatim}
  \end{center}
 \caption{Examples of `bio:tool' metadata for the tools BLAST and ClustalW2}
 \label{fig:biotool-example}
\end{figure}
