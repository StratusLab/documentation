\section{Bioinformatics Appliance Metadata}
\label{sec:bioinfo}

Additionally to the two reference cloud infrastructures, a cloud infrastructure and a Marketplace\@ instance devoted to Biology has been deployed at the
Institut de Biologie et Chimie des Prot\'eines (IBCP) in Lyon (IDB
cloud~\cite{idbcloud}). Operating a separate Marketplace\@ aims to satisfy the availability, data protection and appliance evolution requirements as described in chapter VIII. Indeed, adding a marketplace devoted to the bioinformatics cloud has improved its availability by making it independent from the unavoidable unavailabilities of the central one. An other goal of this separate Marketplace is to be dedicated to the registration of biological appliances, and also because some bioinformatics appliances required to limit their visibility with reasons like, for example, the confidentiality linked to the related scientific projects. Moreover, a fact about some bioinformatics tools is that they may not follow some computing development best practices and then could lead to a strong impact on the creation process of related appliances in terms of number of intermediate appliances and verbosity in the Marketplace\@. 

Biologists and bioinformaticians simultaneously use many of the
bioinformatics tools from the thousands available in the international
community. Frequently they also need to combine multiple software
packages to study their data with either public, or their own,
analysis pipelines. Helping scientists to easily identify suitable appliances
that could assist in the analysis of their data is essential,
especially in the context of federated clouds. 

In such environments, appliances can be created by different experts and could be available
in several places. Bioinformatics experts act as Creators and Endorsers to define virtual
appliances with different bioinformatics tools and workflows. Scientists, as Users, can
then deploy them, on demand, on a research infrastructure. So having additional appliance metadata helps users
find appropriate appliances. We have extended the generic appliance metadata schema (Fig.~\ref{fig:metadata-example})
with additional elements related to bioinformatics tools (Fig.~\ref{fig:biotool-schema}). These metadata can be used to select the
suitable bioinformatics appliance according to the required tools, for example to identify appliances containing the tools BLAST or ClustalW2; or to filter the appliances according to the type of bioinformatics analysis to perform, in this case respectively sequence similarity search or multiple sequence alignment.

Bioinformatics experts, as appliance creators, pre-install the bioinformatics tools related to
the analysis targeted by the new appliance. For example, to create a bioinformatics appliance related to gene or protein sequences analysis, bioinformaticians install BLAST and ClustalW2 in the virtual machine. Once the new appliance is saved, they described these tools by appending the related `bio:tool' descriptions in the appliance metadata (see Fig.~\ref{fig:biotool-exemple}). These
bioinformatics-enhanced metadata are registered in the
Marketplace\@. Bioinformaticians and biologists can then search
the Marketplace with a SPARQL query to find an appropriate appliance containing these  tools, BLAST or ClustalW2 to do multiple sequence alignment of genes or proteins. To help bioinformaticians and biologists to avoid using SPARQL, which is a low level language, we linked the Marketplace\@ with the bioinformatics Web portal we have developed in front of the IDB cloud. On this portal we synchronize, yet manually, the list of suitable appliances  from the Marketplace\@ and provide users with popup menus to filter these appliances according to the above criteria. Finally running the desired bioinformatics appliance is easy
as for any cloud appliance. 


\begin{figure}
\tiny
 \begin{center}
    \begin{verbatim}
<bio:tool rdf:parseType="Resource">
  <dcterms:identifier>tool-name</dcterms:identifier>
  <dcterms:description>
    tool description
  </dcterms:description>
  <slterms:version>tool.version</slterms:version>
</bio:tool>
    \end{verbatim}
  \end{center}
 \caption{Schema of the `bio:tool' metadata}
 \label{fig:biotool-schema}
\end{figure}


\begin{figure}
\tiny
 \begin{center}
    \begin{verbatim}
<bio:tool rdf:parseType="Resource">
  <dcterms:identifier>blast+</dcterms:identifier>
  <dcterms:description>
    BLAST: sequence similarity search
  </dcterms:description>
  <slterms:version>2.2.27</slterms:version>
</bio:tool>
<bio:tool rdf:parseType="Resource">
  <dcterms:identifier>clustalw2</dcterms:identifier>
  <dcterms:description>
    ClustalW2: multiple sequence alignment
  </dcterms:description>
  <slterms:version>2.1</slterms:version>
</bio:tool>
    \end{verbatim}
  \end{center}
 \caption{Examples of `bio:tool' metadata for the tools BLAST and ClustalW2}
 \label{fig:biotool-exemple}
\end{figure}
