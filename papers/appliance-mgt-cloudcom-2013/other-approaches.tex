\section{Existing Appliance Management Techniques}
\label{sec:other-approaches}

Based on the previous generic use cases, it is clear that all cloud
distributions must provide tools or mechanisms for the:
\begin{itemize}
\item Generation of appliances,
\item Storage of appliances,
\item Efficient appliance transport, and
\item Management of appliance metadata.
\end{itemize}
This section describes the approaches used to implement these features
in StratusLab and other cloud distributions.

\subsection{Appliance Factories (Generation)}

As discussed earlier, the manual creation of portable appliances
remains an error-prone, time-consuming process.  In response to this,
it is not surprising that tools and services have appeared to simplify
this task.

Services like Bitnami~\cite{bitnami} are appliance factories that
provide pre-built, ready to deploy appliances.  These can be used as
is, or treated as base appliances that a user can customize with their
own software and services.

There are, however, a larger number of command line tools that
automate the process of creating an appliance. In general these follow
a common pattern of installing a specified base operating system on a
newly created disk image, then customizing that image. The
customization methods vary, using pre-defined, user-generated
templates or command line options.  Examples of these tools include
VMBuilder~\cite{vmbuilder}, VeeWee~\cite{veewee}, and
BoxGrinder~\cite{boxgrinder}.

StratusLab follows this common pattern.  ``Image recipes'' automate
the creation of minimal appliances for most common Linux
distributions, using the distribution's standard configuration
mechanism (e.g. Kickstart for CentOS).  These recipes are used to
update automatically the base StratusLab appliances, which can
subsequently be customized by users.

The StratusLab client includes a tool that automates the customization
of the base appliances.  The tool installs packages specified by the
user and executes a user-defined script to configure the newly
generated appliance. The new appliance is made available in the
StratusLab storage service.

\subsection{Appliance Repositories (Storage)}

No matter what cloud is used, a copy of the appliance contents must
exist on the cloud before virtual machine instances of that appliance
can be started.  Cloud distributions store appliances in a variety of
ways.

The OpenStack project provides appliance discovery, registration and
delivery via its ``Glance'' service. The appliances can be stored in a
simple filesystems or object-storage systems like ``Swift''\@. Both
metadata about registered appliances and the appliances themselves are
exposed via the Glance API\@.

Eucalyptus~\cite{eucalyptus}, an open-source IaaS cloud distribution,
provides an Amazon S3 interface to its ``Walrus'' storage
service. Virtual machine images are stored/retrieved using HTTP
put/get.

OpenNebula uses the concept of a ``Datastore'' for storing appliances.
Multiple datastores can be created backed by one of a selection of
supported filesystem types.  The method used to store and retrieve
appliances depends on the type of datastore used (e.g. filesystem,
iSCSI, Ceph). Sets of ``transfer manager'' scripts that handle the
interaction with the storage backend are provided for each of the
types.

For StratusLab, appliances can be stored in any web accessible
location. The location of the appliance is contained in the metadata
published in the Marketplace\@. This concept makes it possible to
share appliances between StratusLab users, and also between users of
different cloud infrastructures, because of the open accessibility and
portability of StratusLab appliances
(Sec.~\ref{sec:portable-appliances}).

\subsection{Appliance Transport}

\label{appliance-transport}
In a federated cloud environment, the appliances usually are made
available outside of a particular cloud infrastructure and must be
transported to the cloud before use.  Most cloud distributions require
the user to do this manually, but some tools exist to facilitate
automated transfers.

The vmcaster/vmcatcher tools developed within the HEPiX Virtualization
Working Group ~\cite{hepixbooktransfer}, use the concept of
subscriptions to an appliance list. It makes the download and
transport of an appliance from the appliance list similar to using a
system package manager.  The downloaded appliances are verified
against their X509 signatures and cached.

In StratusLab, appliances are transported from a web server or from
cloud storage. Based on the appliance identifier in the Marketplace,
the transport of the appliance is done transparently by the cloud
infrastructure.  The downloaded appliances are then verified and
cached in the persistent disk storage, ensuring that the transport of
a given appliance is done only once.

Making the transfers transparent to users, requires that these tools
be integrated with the cloud's appliance management workflows.

\subsection{Appliance Registry}

An appliance registry allows users to search for existing appliances
based on certain criteria.  Any cloud distribution that allows users
to share appliances must have a registry of some sort. Most
distributions do provide such a service, although in many cases it is
integrated with the appliance repository.

The Eucalyptus Image Store provides a set of base appliances that can
be downloaded and imported into a local Eucalyptus cloud using a
command-line client.

The OpenNebula AppMarket allows registered appliance developers to
upload appliances that users can then download to use in their local
cloud. This is done using the OpenNebula command-line client, or
through the ``Sunstone'' GUI.

The EGI Applications Database~\cite{AppDB}, AppDB, is a central service
where appliance lists are published.   
Vmcaster/vmcatcher (Sec.~\ref{appliance-transport}) tools are 
integrated in AppDB, making possible producing, signing, and publishing 
appliance lists.
Each resources provider subscription to an appliance list is published
in the AppDB, and appliances are automatically distributed to all those 
resources provider that subscribed to the respective appliance list. 
 
The StratusLab Marketplace is at the center of the appliance handling
mechanisms in the StratusLab cloud distribution. It contains metadata
about appliances and serves as a registry for shared appliances.  In
order to use and/or share an appliance, its metadata must be
registered in the Marketplace\@.

Once an appliance is created, StratusLab provides simple tools for
building, cryptographically signing with a valid certificate, and
uploading the metadata to the Marketplace\@.  The Marketplace
validates the metadata entry and verifies the email address of the
endorser.  If all the checks pass, the metadata will then be visible
in the Marketplace and other users can search for the entry.
