\section{Planned Improvements}
\label{sec:production}

In parallel with its software development, the collaboration operates
a federated cloud infrastructure with sites in Orsay, France and
Athens, Greece\@.  These sites share a common user authentication
framework and Marketplace allowing users to allocate resources and to
use appliances on either site.  The collaboration uses this production
cloud to validate its software in real world conditions.

\begin{table}
\caption{Marketplace Statistics}
\label{tab:statistics}
\begin{center}
\begin{tabular}{lr}
\hline
\hline
endorsers & 44 \\
current appliances & 105 \\
deprecated appliances & 170 \\
expired appliances & 832 \\
\hline
\hline
\end{tabular}
\end{center}
\end{table}

As a core service, the Marketplace is heavily used by both members of
the collaboration and users of the federated cloud.  Some statistics
on its use can be found in Table~\ref{tab:statistics}.  Overall the
service has performed well, with minor problems being addressed as the
software evolves over time.  Some outstanding issues and potential
solutions are described below.

\subsection{Availability}

Having a central Marketplace instance allows users to easily find all
of the appliances from a single location.  Similarly, it allows image
creators to upload the metadata just once.  However, the Marketplace
is consulted every time a new machine instance is launched to check if
an appliance has been deprecated.  Consequently if the Marketplace is
not available, new instances cannot be created on any cloud relying on
the Marketplace\@.  Future iterations of the Marketplace must provide
redundancy and high-availability of the Marketplace service. 

To provide for this, a replication scheme will be implemented that
allows for multiple Marketplace instances to be deployed, each
maintaining a local copy of the metadata entries. As all the
information required to rebuild the metadata index stored in Sesame is
the set of raw metadata files, it is only these that need be
replicated.  A potential solution would be to use a Git repository as
the core `database' for the metadata entries, with each Marketplace
updating its index periodically from a local clone of the global
repository.

\subsection{Data Protection}

By design the appliance metadata is considered public.  In reality,
however, there are both users and administrators that would like to
restrict the visibility of the appliance metadata.  Many cloud
administrators would like to run a ``private'' Marketplace to limit
the visibility of certain appliances while still taking advantage of
the central, public Marketplace instance.  There must be a mechanism
for federating Marketplace instances in the future and the move to
using Git for metadata file management may also facilitate the
federation of different Marketplace instances.

\subsection{Appliance Quality}

Although the metadata contains a significant amount of information
about an appliance, it does not contain information about how well the
appliance functions for users.  A common request has been to add
social features to the Marketplace to allow users of an appliance to
leave comments and to signal problems with the appliance itself.  A
possible approach to add these features without overly complicating
the Marketplace implementation would be to make use of an external
service such as Disqus~\cite{disqus}.

\subsection{Appliance Evolution}

Appliances naturally evolve as operating system updates are applied
and new services are added.  However each time an appliance is
updated, the SHA-1 hash and the corresponding image identifier change.
This makes it difficult for users to track the evolution of an image
and impossible to use a stable identifier for, for example, the latest
version of the CentOS image.

Recently a `tag' feature has been added to the Marketplace\@.  This
allows an endorser to provide a simple label for a series of images,
where the tag (namespaced by the endorser's email) will always resolve
to the latest image identifier.  By using the tag, users can always
use the latest version of an image without having to look up manually
the associated identifier. 

A complementary and more rigorous solution would be to make use of the
Dublin Core terms \emph{replaces} and \emph{isReplacedBy}. This would
provide a link in each metadata entry to the previous and next entries
in the evolution of the appliance.  The StratusLab tools must be
updated to simplify the use of these terms to ensure that they are
widely used.
