\section{Marketplace Image Management}
\label{sec:approach}

{\em Start with what is different about the Marketplace: emphasis on
  sharing of images (especially in federated environments), adaptable
  to different domains (i.e. flexible metadata can be specialized,
  trust mechanisms are open to actors), and being neutral with respect
  to cloud implementations.  Describe very high level design of system
  and interactions with the actors and cloud services.  The technical
  note was organized around specific workflows; this may be good to
  keep for the paper.  The section should explicitly mention that
  ``third-party'' endorsers could be supported in this system.}

The Marketplace it at the center of the image handling mechanisms in
the StratusLab cloud distribution.  It contains metadata about images
and serves as a registry for shared images.  There are two primary use
cases for images in the cloud: 1) creating new images and making them
available and 2) instantiating a referenced image.

\subsection{Registering a New Image}

New machine or disk images can either be created from scratch
following the StratusLab guidelines or be modified versions of
existing images.  The workflow for creating an image and making it
available for other involves the following steps:

\paragraph{Image Creation} A new machine or disk image is created from
  scratch or from an existing image.  The \cmd{stratus-create-image}
  command can facilitate this process.

\paragraph{Transfer to Storage} The new image is transferred to network
  accessible storage.  This may be storage provided by a StratusLab
  cloud or storage provided by the user or users' institute.  The
  physical image must be available via one or more URLs\@.

\paragraph{Create Metadata} The metadata entry for the image must be
  created.  This is an XML file containing information about the image
  itself including the location URLs for the physical image.  The
  \cmd{stratus-build-manifest} command may be helpful here.

\paragraph{Sign Metadata} All of the metadata entries must be signed with a
  valid grid certificate.  The \cmd{stratus-sign-metadata} will sign
  the metadata file with a given certificate, inserting the
  certificate information automatically into the metadata file.

\paragraph{Upload Metadata} The signed metadata entry can then be uploaded
  to the StratusLab Marketplace\@.  This is done via an HTTP POST to a
  the Marketplace's \url{http://mp.example.org/metadata/} URL\@.

The Marketplace will then validate the entry and verify the email
address given in the image metadata.  Once all of the checks have been
done, the metadata will be visible in the Marketplace and other users
can search for the entry.

\subsection{Using a Registered Image}

Referencing an existing image and instantiating it involves the
following steps:

\paragraph{Request New Instance} StratusLab users will create new machine
  instances through the \cmd{stratus-run-instance} command.  Disk
  images can be used in the machine description to attach existing,
  public disks to a new instance.  In both cases, the user can supply
  an image identifier (e.g. ``MMZu9WvwKIro-rtBQfDk4PsKO7\_''), a
  Marketplace URL
  (e.g. \url{http://mp.example.org/metadata/?id=MMZu9WvwKIro-rtBQfDk4PsKO7\_},
  any URL that provides (directly or indirectly) a valid metadata
  entry (e.g. \url{http://tinyurl.com/my-favorite-image}), or the
  physical location of the image
  (e.g. \url{http://example.org/myimage}).

\paragraph{Resolve Entry} The URL given by the user must be resolved into
  one or more metadata entries associated with the image.  For any of
  the URLs that provide the metadata entry (or entries) directly, this
  is a simple HTTP GET on the URL\@.  For location based URLs, this
  will involve a search in the Marketplace either on the location URL
  or on a checksum calculated on the fly.

\paragraph{Apply Site Policy} The command \cmd{stratus-policy-image} will
  apply a configurable site policy to the image metadata.  If none of
  the metadata entries for a particular image pass the site policy,
  then the image will not be instantiated.

\paragraph{Download} Assuming that the image passed the site policy, then
  the \cmd{stratus-download-image} can be used to download a copy of
  the image based on the location URLs in the metadata entry.

Once the image is available, it will be instantiated and run.  All of
the work on the cloud side, it expected to happen in the clone hook of
OpenNebula\@.
