\section{Bioinformatics Appliance Metadata}
\label{sec:bioinfo}

Biologists and bioinformaticians simultaneously use many of the
bioinformatics tools from the thousands available in the international
community. Frequently they also need to combine multiple software
packages to study their data with either public, or their own,
analysis pipelines. Helping scientists to easily identify appliances
that could assist in the analysis of their data is essential,
especially in the context of federated clouds. In such environments
appliances can be created by different experts and could be available
in several places.  Having extensible appliance metadata helps users
find appropriate appliances.

A Marketplace instance devoted to biology has been deployed at the
Institut de Biologie et Chimie des Prot\'eines (IBCP) in Lyon (IDB
cloud~\cite{idbcloud}). Here, bioinformatics experts define virtual
appliances with pre-installed tools and workflows, that scientists can
then deploy, on demand, on national research infrastructures.

As identifying and running the relevant bioinformatics appliances
could be difficult, IBCP have extended the standard appliance metadata
with additional elements, where bioinformatics tools are referenced
and tagged with RDF metadata. These metadata can be used to select the
right bioinformatics appliance according to the desired tools
(\textit{e.g.} BLAST, ClustalW2) or the type of analysis to perform
(\textit{e.g.} sequence or structural analysis).

Bioinformatics experts pre-install the bioinformatics tools related to
the analysis to be targeted by the new appliance. The `biotools' are
described in the appliance metadata as RDF resources (see
Fig.~\ref{fig:biotool-schema}), inserted into a complete description
like that in Fig.~\ref{fig:metadata-example}, and this
bioinformatics-enhanced metadata is registered in the
Marketplace\@. Bioinformaticians and biologists are then able to query
the Marketplace for an appliance containing a particular tool, for
example ClustalW2 to do multiple sequence alignment of genes or
proteins. Finally running the desired bioinformatics appliance is easy
as for any cloud appliance.

\begin{figure}
\tiny
 \begin{center}
    \begin{verbatim}
<bio:biotool rdf:parseType="Resource">
  <dcterms:identifier>clustalw2</dcterms:identifier>
  <dcterms:description>
    ClustalW2: multiple sequence alignment
  </dcterms:description>
  <slterms:version>2.1</slterms:version>
</bio:biotool>
    \end{verbatim}
  \end{center}
 \caption{Example `biotool' metadata}
 \label{fig:biotool-schema}
\end{figure}
