\section{Design and Requirements}

\subsection{StratusLab}

The StratusLab collaboration provides a complete, open-source solution
for deploying public or private ``Infrastructure as a Service'' (IaaS)
cloud infrastructures and is designed to be both simple to install and
simple to use.~\cite{slbook}

In parallel, the collaboration operates a federated cloud
infrastructure with two sites located in Orsay, France and Athens,
Greece\@.  These sites share a common user authentication framework
and appliance registry (the ``Marketplace'') allowing users to
allocate resources and to use appliances on either site.  The
StratusLab collaboration uses this production cloud to validate its
software in real world conditions.

\subsection{Marketplace}

The primary requirement of the Marketplace is that it permits all
users to search the range of available images for ones which satisfy
their requirements--avoiding the effort to create a customized image.
Equally important, the Marketplace provides a repository of image
metadata that can be used by cloud administrators to decide if a
particular image is trusted and can be run on their cloud.

As described above, the StratusLab cloud distribution provides
mechanisms for storage and retrieving the appliance files themselves.
A conscious design decision was made to separate the storage and
transport of appliance file from the Marketplace implementation.  (A
feature not shared by all cloud distributions.)  Storing the image
files outside of the Marketplace:
\begin{itemize}
\item Makes it easier to scale the Marketplace implementation and to
  create mirrors of it.
\item Allows owners of the image to control access to the image
  itself.
\item Makes the Marketplace implementation independent of the
  transport protocol, allowing many different protocols to be
  supported.
\item Relieves the operator of the Marketplace from concerns related
  to image copyrights.
\end{itemize}
Related tools for downloading referenced images and validation of
those images must complement the Marketplace service.

\subsection{Requirements}

Table~\ref{tab:requirements} contains a detailed list of requirements
for the Marketplace, as an appliance registry, and for the appliance
metadata.  The core requirements have been derived from the primary
use cases described above, taking into account the need to create a
functional service.  

\subsection{Timeline}

A core concept within the Marketplace is the appliance timeline.  The
Marketplace maintains a complete history of all endorsements related
to an image.  Users of the Marketplace can retrieve the full history,
for example to conduct an audit on why a particular image was
authorized.  Normally however, the Marketplace users only want to see
the current endorsements for an appliance, that is the list of the
latest, non-expired endorsements from all of the endorsers of an
image.  This allows users and administrators alike to decide if an
image is currently valid.  

\begin{table}
\caption{Requirements}
\label{tab:requirements}
\begin{center}
\begin{tabular}{p{0.95\columnwidth}}
\hline\hline

\\ Allow anyone to upload valid metadata descriptions to the site.

\\ Valid descriptions must be signed by a grid certificate.  The
  endorser information must match the information in the certificate
  itself.  It may also be desirable to allow signatures with a PGP key
  or SSH key.

\\ Users can ``replace'' existing metadata descriptions by
  uploading a new signed description.  Nonetheless, all validated
  descriptions uploaded to the site must always be available to
  provide a history of the metadata evolution.

\\ Users must be able to search the metadata database on a
  reasonable subset of the possible keys.  Two required keys are the
  image identifier and the endorser's email address.

\\ Users must be able to download the original signed metadata in
  the RDF/XML format from the registry.  This is the only format that
  allows the metadata to be cryptographically verified.

\\ The registry should allow the metadata to be downloaded in
  alternate formats, notably JSON and HTML.

\\ The service must be easy to use from all programming languages
  (including scripting languages) and usable from a web browser.

\\ \hline 
\\
  Descriptions of available images should contain at least one
  location from which the image can be obtained.  Descriptions without
  a location are appropriate only if the image becomes unavailable.

\\ Valid descriptions must contain a valid email address.  The
  service must confirm the email address for every upload of metadata.

\\ All valid descriptions must contain a creation date for the
  endorsement.  The server must only accept descriptions with a
  creation date more recent that the current latest.

\\ There may be several sets of metadata associated with a
  particular machine or disk image. This allows third parties to
  endorse images created by someone else. (E.g. VO endorses an image
  created by StratusLab.)

\\ The underlying schema for the metadata descriptions must be
  flexible and extensible.  The accounts for differing needs of users
  and eventual evolution of those needs.

\\
\hline\hline
\end{tabular}
\end{center}
\end{table}
