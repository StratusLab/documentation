\section{Design and Requirements}

{\em Dive into more detail about the design and various requirements
  for the system.}

The primary requirement of the Marketplace is that it permits all
users to search the range of available images for ones which satisfy
their requirements--avoiding the effort to create a customized image.
Equally important, the Marketplace provides a repository of image
metadata that can be used by cloud administrators to decide if a
particular image is trusted and can be run on their cloud.

Although the StratusLab cloud distribution must provide a mechanism
for storing and retrieving the image file themselves, a conscious
decision was made to separate the storage and transport of image file
from the Marketplace implementation.  Storing the image files outside
of the Marketplace:
\begin{itemize}
\item Makes it easier to scale the Marketplace implementation and to
  create mirrors of it.
\item Allows owners of the image to control access to the image
  itself.
\item Makes the Marketplace implementation independent of the
  transport protocol, allowing many different protocols to be
  supported.
\item Relieves the operator of the Marketplace from concerns related
  to image copyrights.
\end{itemize}
Related tools for downloading referenced images and validation of
those images must complement the Marketplace service.

\begin{table}
\caption{Requirements}
\label{tab:requirements}
\begin{tabular}{p{0.4\textwidth}}
\hline\hline
Allow anyone to upload valid metadata descriptions to the site.

\\ Valid descriptions must be signed by a grid certificate.  The
  endorser information must match the information in the certificate
  itself.  It may also be desirable to allow signatures with a PGP key
  or SSH key.

\\ Valid descriptions must contain a valid email address.  The
  service must confirm the email address for every upload of metadata.

\\ All valid descriptions must contain a creation date for the
  endorsement.  The server must only accept descriptions with a
  creation date more recent that the current latest.

\\ There may be several sets of metadata associated with a
  particular machine or disk image. This allows third parties to
  endorse images created by someone else. (E.g. VO endorses an image
  created by StratusLab.)

\\ Users can ``replace'' existing metadata descriptions by
  uploading a new signed description.  Nonetheless, all validated
  descriptions uploaded to the site must always be available to
  provide a history of the metadata evolution.

\\ Users must be able to search the metadata database on a
  reasonable subset of the possible keys.  Two required keys are the
  image identifier and the endorser's email address.

\\ Users must be able to download the original signed metadata in
  the RDF/XML format from the registry.  This is the only format that
  allows the metadata to be cryptographically verified.

\\ The registry should allow the metadata to be downloaded in
  alternate formats, notably JSON and HTML.

\\ Descriptions of available images should contain at least one
  location from which the image can be obtained.  Descriptions without
  a location are appropriate only if the image becomes unavailable.

\\ The service must be easy to use from all programming languages
  (including scripting languages) and usable from a web browser.

\\ The underlying schema for the metadata descriptions must be
  flexible and extensible.  The accounts for differing needs of users
  and eventual evolution of those needs.

\\

\hline\hline
\end{tabular}
\end{table}
