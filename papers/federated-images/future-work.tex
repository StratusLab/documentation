\section{Future Work}

Description of the current issues and how the next implementation will
deal with them. 

Points for improvement:
\begin{itemize}
\item Difficult to determine the quality of listed images.  Croudsource
  this with something like DISCUS?
\item Single point of failure.  Need to allow for replication.  (Git as
  storage mechanism?)
\item Need for public/private instances (Git as management backend to
  merge different repositories?)
\item SPARQL queries are essentially unused by users (but used
  internally).  Remove to simplify code?
\item Need to have fixed URL for latest in a series of images.
\item Allow 'prettier' descriptions of the images in the Marketplace.
\end{itemize}

Currently the Marketplace is a single point of failure in the StratusLab infrastructure. This is 
a critical issue that is to be addressed in upcoming releases. A replication scheme will be 
implemented to allow multiple Marketplace instances to be run, each maintaining its own local
copy of the global metadata database. As all the information required to rebuild the metadata index
stored in Sesame is the set of raw metadata files, it is only these that need be replicated. A potential solution 
to provide this replication is to use a git repository as the core `database' for the metadata entries.  
For a public Marketplace, the repository could be hosted on GitHub~\cite{github} with each replicated server 
pulling/pushing from that repository. This only requires that a Marketplace database be updated periodically 
from a local clone of the global repository.  For a private Marketplace, the git repository can exist on the local disk.
An alternative solution is to use a distributed database, such as CouchBase~\cite{couchbase}, to maintain the
  global database. Updates of replica instances would be handled periodically in much the same way as with Git.