\section{Future Work}
\label{sec:future-work}
As described above the Marketplace currently represents a single point of failure in the StratusLab infrastructure. This is a critical issue that is to be addressed in upcoming releases. A replication scheme will be implemented to allow multiple Marketplace instances to be run, each maintaining its own local copy of the global metadata database. As all the information required to rebuild the metadata database stored in Sesame is the set of raw metadata files, it is only these that need be replicated. A potential solution to provide this replication is to use a Git repository as the core `database' for the metadata entries. This only requires that a Marketplace database be updated periodically from a local clone of the global repository. 

An added benefit of this approach is that it would allow for the public/private Marketplace requirment described in Section~\ref{sec:production}. For a public Marketplace, the repository could be hosted on a public service, such as GitHub~\cite{github} with each replicated server pulling/pushing from that repository.  For a private Marketplace, the repository can exist only on the local disk. An alternative solution could be to use a distributed database, such as CouchBase~\cite{couchbase}, to maintain the global database. Updates of replica instances would be handled periodically in much the same way as with Git.

An important ability is to trace the evolution of an image through its metadata. Currently modifying an appliance results in a new identifier being created, which will then be used in the updated metadata. This breaks the link with the previous version. The recently added `tag' functionality goes someway to solving this.  The metadata entries for a particular appliance can be found by retrieving all entries containing a particular tag. This element is optional, however, and may not be added to all entries. A more rigourous solution would be to make use of the Dublin Core terms `replaces' and `isReplacedBy'. This would provide a link in each metadata entry to the previous and next entries in the lifetime of an appliance. Using these terms could add a significant burden to the user creating a metadata entry, so this would have to be automated as much as possible.

Improvements to the interface will include the ``social-features'' mentioned above, along with re-designing the various displays, such as the appliance list and the metadata detailed view. As stated the purpose of the social features would be to allow users to determine the quality of an appliance referred to by a metadata entry. This could be done with a comment, or rating, system. A simple way to add using an external service such as Disqus~\cite{disqus}.

