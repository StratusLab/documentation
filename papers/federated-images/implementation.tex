\section{Implementation}
\label{sec:implementation}

{\em Describe the implementation chosen and the technology choices.}

{\em Description of the technology choices made, especially those that
  affect the outward functionality of the system.}

{\em The metadata schema should be covered along with an explaination
  of why something like OVF was not used.}

\subsection{Technology Choices}

To ensure that the service can be used easily with all programming
languages, the Marketplace exposes a RESTful~\cite{rest} interface
over standard HTTP(S).  This interface is implemented using
RESTlet~\cite{restlet}, a Java framework for RESTful
services.

The user interface to the Marketplace, accessed through a Browser, is implemented
using a combination of FreeMarker~\cite{freemarker}, a Java Template Engine Library, 
JQuery~\cite{jquery}, custom HTML and JavaScript.

Semantic web technologies were designed to manage metadata about
(third-party) resources identified with a URI.  Consequently, they are
ideally suited to this situation in which the Marketplace must manage
metadata about machine and disk images.  These technologies already
provide standard formats for the metadata (RDF/XML~\cite{rdfxml,
  rdfprimer, rdfschema}) and query languages (SeRQL~\cite{serql},
SPARQL~\cite{sparql}).  The Marketplace implementation makes use of the
Sesame~\cite{sesame} framework to provide search
capabilities over the metadata database.

To validate the metadata associated with a particular image, it is
necessary to cryptographically sign individual entries.  As the raw
format used for the metadata entries will be XML, the XML
Signature~\cite{xmlsig} specification is reused.  This is
conveniently a standard part of modern Java runtime environments.

Working with RDF also requires an agreed vocabulary to ensure a common
semantic meaning of the metadata tags.  The Dublin Core Metadata
Initiative has published a vocabulary~\cite{dcterms} that can be used
for much of the image metadata descriptions.  This is complemented by
a vocabulary specific to StratusLab.  Using RDF also allows additional
metadata fields to be specified (in separate namespaces) to complement
the standard fields described in this document.

\subsection{Image Identifiers}

To ensure an unambiguous connection between the metadata entries
contained in the Marketplace and the described machine and disk
images, a unique image identifier must be used that depends solely on
the image's contents.

For the purposes of the Marketplace and caching of images, an
identifier derived from the SHA-1 hash of the image file is used.  The
identifier is the 160-bit SHA-1 hash of the image file (treating the
file as a binary file), extended with two zero bits to the left to
form a 162-bit value.  This 162-bit value is then encoded using the
base64url encoding to produce a 27 character image identifier.  For
example, the SHA-1 hash (written in hex):
\begin{verbatim}
c319bbd5afc0a22ba3eaed0507c39383ec28eeff
\end{verbatim}
becomes the image identifier:
\begin{verbatim}
MMZu9WvwKIro-rtBQfDk4PsKO7_
\end{verbatim}
using this algorithm.  Note that to avoid inadvertent or malicious
collisions of the SHA-1 additional checksum within the metadata
description should also be verified.

\subsection{Verification of Uploaded Metadata}

When new metadata entries are uploaded to the Marketplace, the service
validates the entry before accepting it.  This involves three
separate steps:
\begin{enumerate}
\item The metadata must be a valid RDF/XML file, following the defined
  schemas and conventions.
\item To avoid having old metadata entries being reposted (for example
  to make a deprecated image appear to be valid), the server
  ensures that only entries endorsed after the current entry for a particular
  image are accepted.
\item To ensure that the included email address is valid, all
  additions to the Marketplace must be confirmed by email.
\end{enumerate}
Only validated metadata entries are visible from the standard
Marketplace interface. 

\subsection{REST Resource URLs}

The mapping between URLs and service resources is the central part of
any RESTful service.  The resource mapping must provide convenient
access via a browser but also facilitate automated interactions with
tools.  Table~\ref{table:restmap} provides the URL mapping for the
Marketplace.  (The `delete' and `put' methods are not supported by any
URLs.)  Within the table ``identifier'' refers to the 27 character
image identifier, ``email'' refers to the endorser's email address,
and ``date'' refers to endorsement date written in the format {\tt
  yyyy-MM-ddThh:mm:ssZ}.  All of the URLs support an
XML and HTML formats.  In addition single metadata entries also 
provide a JSON representation.

\begin{table*}
\caption{REST URL Mapping for the Marketplace}
\label{table:restmap}

\begin{tabular}{l*4{p{5cm}}}

\hline
URL & get & post & options \tnl
\hline
\hline

/ & 
landing page with information about service and internal links  & 
X  & 
last update time  \tnl 
\hline
\hline

/slreq &
list of metadata terms in slreq namespace &
X &
X \tnl
\hline

/slreq/\replaceable{name} &
description of particular term in slreq namespace &
X &
X \tnl
\hline
\hline

/slterms &
list of metadata terms in slterms namespace &
X &
X \tnl
\hline

/slterms/\replaceable{name} &
description of particular term in slterms namespace &
X &
X \tnl
\hline
\hline

/endorsers &
list of endorsers in database &
X &
number, last update time \tnl
\hline

/endorsers/\replaceable{email} &
statistics about particular endorser &
X &
number, last update time \tnl
\hline
\hline

/metadata/?\replaceable{query} & 
list of image identifiers and selected fields (query terms of identifier, email, and created are supported to refine list)  & 
add another metadata entry  & 
number, last update time  \tnl 
\hline

/metadata/\replaceable{identifier}/\replaceable{email}/\replaceable{date}  & 
unique metadata entry  & 
X  & 
X  \tnl 
\hline

/metadata/query  & 
form for simple query of service  & 
submission of complete query document  & 
X  \tnl 
\hline
\hline

/confirmations/\replaceable{UUID}?action=\replaceable{action} & 
confirmation of upload  (action can be 'CONFIRM', 'ABORT', or 'ABUSE') & 
X  & 
X  \tnl 
\hline
\hline

\end{tabular}
\end{table*}

\subsection{Storage and Query of Metadata}

The metadata entries are stored using the Sesame~\cite{jena}
interfaces.  This allows simple queries and queries with SPARQL~\cite{sparql}
 to be easily supported.

Unfortunately, Sesame is unable to accept the signed metadata.
Consequently, the raw, signed metadata files are saved on
a file system as well.  These are returned directly when XML is
requested.


