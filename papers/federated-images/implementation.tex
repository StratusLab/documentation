\section{Implementation}
\label{sec:implementation}

The Marketplace implementation uses standard web technologies to
create a service accessible programmatically and via a web browser.
For programmatic access, the service exposes an interface over HTTP(S)
using RESTlet~\cite{restlet}, a Java framework for RESTful~\cite{rest}
services.  HTML representations and browser interactions are provided
with a combination of FreeMarker~\cite{freemarker} (a Java template
engine library), CSS, JavaScript, and JQuery~\cite{jquery}.

\subsection{Identifiers}

The separation of the appliance metadata and the appliance contents
requires an unambiguous mechanism for matching the two.  StratusLab
uses the SHA-1 hash of the appliance contents to generate an
unambiguous, intrinsic identifier for the image.  The identifier is
the 27 character string generated by encoding the SHA-1 checksum with
the base64url encoding.

\subsection{Metadata}

\begin{figure}
\begin{center}
\tiny
\begin{verbatim}
<?xml version="1.0" encoding="UTF-8" standalone="no"?>
<rdf:RDF
  xmlns:dcterms="http://purl.org/dc/terms/"
  xmlns:rdf="http://www.w3.org/1999/02/22-rdf-syntax-ns#"
  xmlns:slreq="http://mp.stratuslab.eu/slreq#"
  xmlns:slterms="http://mp.stratuslab.eu/slterms#"
  xml:base="http://mp.stratuslab.eu/">
  
  <rdf:Description rdf:about="#G2yRqidaqqJ0mUB6UKR-26eeiJA">

    <dcterms:identifier>G2yRqidaqqJ0mUB6UKR-26eeiJA</dcterms:identifier>

    <slreq:bytes>5368709120</slreq:bytes>

    <slreq:checksum rdf:parseType="Resource">
      <slreq:algorithm>MD5</slreq:algorithm>
      <slreq:value>9eb3...</slreq:value>
    </slreq:checksum>
    <slreq:checksum rdf:parseType="Resource">
      <slreq:algorithm>SHA-1</slreq:algorithm>
      <slreq:value>6db2...</slreq:value>
    </slreq:checksum>
    <slreq:checksum rdf:parseType="Resource">
      <slreq:algorithm>SHA-256</slreq:algorithm>
      <slreq:value>d3d3...</slreq:value>
    </slreq:checksum>
    <slreq:checksum rdf:parseType="Resource">
      <slreq:algorithm>SHA-512</slreq:algorithm>
      <slreq:value>b122...</slreq:value>
    </slreq:checksum>
    
    <slreq:endorsement rdf:parseType="Resource">
      <dcterms:created>2013-05-17T10:12:22Z</dcterms:created>
      <slreq:endorser rdf:parseType="Resource">
        <email>images@stratuslab.eu</email>
        <subject>CN=StratusLab,OU=Create Images,O=StratusLab Project,C=EU</subject>
        <issuer>CN=StratusLab,OU=Create Images,O=StratusLab Project,C=EU</issuer>
      </slreq:endorser>
    </slreq:endorsement>

    <dcterms:type>base</dcterms:type>
    <slterms:kind>machine</slterms:kind>
    
    <slterms:os>CentOS</slterms:os>
    <slterms:os-version>6.4</slterms:os-version>
    <slterms:os-arch>x86_64</slterms:os-arch>
    <slterms:version>1.0</slterms:version>
    <dcterms:compression>gz</dcterms:compression>
    <slterms:location>http://appliances.stratuslab.eu/...</slterms:location>
    
    <dcterms:format>raw</dcterms:format>
    
    <dcterms:creator>StratusLab</dcterms:creator>
    
    <dcterms:created>2013-05-12T20:54:55Z</dcterms:created>
    <dcterms:valid>2013-11-06T20:54:55Z</dcterms:valid>
    
    <dcterms:title/>
    <dcterms:description>CentOS 6.4 base image...</dcterms:description>
    
    <slterms:hypervisor>kvm</slterms:hypervisor>
    <slterms:disks-bus>virtio</slterms:disks-bus>
    
    <dcterms:publisher>StratusLab</dcterms:publisher>

    <!--
      Users may add additional, namespaced metadata here. 
    -->    
    
  </rdf:Description>
  <Signature xmlns="http://www.w3.org/2000/09/xmldsig#">
  ...
  </Signature>
</rdf:RDF>
\end{verbatim}
\end{center}
\caption{Abbreviated CentOS Appliance Metadata Description}
\label{fig:metadata-example}
\end{figure}


Semantic web technologies were designed to manage metadata about
(third-party) resources identified with a URI\@.  Consequently, they
are ideally suited to this situation in which the Marketplace must
manage metadata about appliances.  These technologies already provide
standard formats for the metadata (RDF/XML~\cite{rdfxml, rdfprimer,
  rdfschema}) and query languages (SeRQL~\cite{serql},
SPARQL~\cite{sparql}).  The Marketplace implementation makes use of
the OpenRdf Sesame~\cite{sesame} framework to provide search
capabilities over the metadata database.

Working with RDF also requires an agreed vocabulary to ensure a common
semantic meaning of the metadata tags.  The Dublin Core Metadata
Initiative has published a vocabulary~\cite{dcterms} that can be used
for much of the appliance metadata descriptions.  This is complemented
by a vocabulary specific to StratusLab\@.  Using RDF also allows
additional metadata fields to be specified (in separate namespaces) to
complement the standard fields. RDF with Dublin Core maintains a good
balance between machine and human readability. (See
Fig.~\ref{fig:metadata-example} for an abbreviated example for a
CentOS appliance.)

As the overall aim is to provide a high-level description of an
appliance these metadata standards are more suited than something more
heavy-weight, such as the Open Virtualization Format
(OVF~\cite{ovf}). OVF describes the packaging and distribution of a
full virtual machine rather than just an appliance, and so would
contain a large volume of additional information that is not
particularly relevant for cloud users and administrators. It should be
noted however, that the RDF metadata descriptions are easily
extensible and could include the OVF metadata if necessary.
Similarly, the use of OVF to package the appliance itself is not
precluded.

To validate the metadata associated with a particular appliance, it is
necessary to sign individual entries cryptographically.  As the raw
format used for the metadata entries is XML, the XML
Signature~\cite{xmlsig} specification is reused.  Conveniently, modern
Java runtime environments include this as a standard part of the
API\@.

\subsection{REST Resource URLs}

The mapping between URLs and service resources is the core
specification of any RESTful service.  The resource mapping must
provide convenient access via a browser but also facilitate automated
interactions with tools.  Table~\ref{table:restmap} provides the URL
mapping for the Marketplace along with the actions associated with the
given HTTP verbs.  (The `delete' and `put' methods are not supported
by any URLs.)  Within the table ``identifier'' refers to the 27
character image identifier, ``email'' refers to the endorser's email
address, and ``date'' refers to endorsement date written in the format
{\tt yyyy-MM-ddThh:mm:ssZ}.  All of the URLs support XML and HTML
representations.  Individual metadata entries also provide a JSON
representation.

\begin{table}
\caption{Core REST Resources}
\label{table:restmap}
\begin{center}
\begin{tabular}{ll}

\hline
\hline

\multicolumn{2}{l}{{\tt /}} \tnl
GET & redirects to /metadata resource \tnl
\hline

\multicolumn{2}{l}{{\tt /endorsers}} \tnl
GET & list of endorsers in database \tnl
OPTIONS & number of endorsers; last update time \tnl
\hline

\multicolumn{2}{l}{{\tt /endorsers/\replaceable{email}}} \tnl
GET & statistics about particular endorser \tnl
OPTIONS & number of entries; last update time \tnl
\hline

\multicolumn{2}{l}{{\tt /metadata/?\replaceable{query}}} \tnl
GET & list of identifiers and selected fields (query terms of \tnl
    & (identifer, email, and created can be used to refine list) \tnl
POST & create new metadata entry \tnl
OPTIONS & number of entries; last update time \tnl
\hline

\multicolumn{2}{l}{{\tt /metadata/\replaceable{identifier}/\replaceable{email}/\replaceable{date}}} \tnl
GET & unique metadata entry \tnl
\hline

\multicolumn{2}{l}{\tt /query} \tnl
GET & form for simple query of service \tnl
POST & submit query \tnl
\hline

\multicolumn{2}{l}{\tt /upload} \tnl
GET & form for browser upload of metadata entry \tnl
POST & create new entry via post to /metadata \tnl

\hline
\hline

\end{tabular}
\end{center}
\end{table}

\subsection{Storage and Query of Metadata}

Two copies of successfully validated and confirmed metadata are stored
on the filesystem. The original uploaded file is saved unmodified,
while a second copy stripped of the XML signature is added to the
Sesame RDF repository. Storing the metadata in the repository allows
both simple SPARQL~\cite{sparql} queries to be easily supported.  A
request for a specific metadata entry in XML format returns the
original signed file.
