\section{Introduction}
\label{sec:Introduction}

Cloud technologies are revolutionizing the way in which centralized
computing infrastructures are exposed to and used by scientists and
engineers.  These technologies enable rapid provisioning, dynamic
scaling, and environment customization that was impossible to achieve
previously. 

In particular, the ability to customize fully the execution
environment has drastically increased the utility of shared computing
infrastructures.  Users can create ``appliances'' that have their
desired operating system and required applications pre-installed
and pre-configured for their needs, independently of the needs of the
other users.

Being able to create, use, and reuse such appliances eliminates the
need for multiple installations of applications, the costs associated
with porting and re-validation of applications in varying envionments,
and errors associated with running in different environments, together
allowing users to concentrate on their calculations rather than the
software.

Some of these reductions in effort, however, are offset by the need to
manage the appliances themselves.  Tools for this appliance management
vary greatly between cloud distributions and few are intended to be
used in federated cloud environments.

\subsection{StratusLab}

The StratusLab collaboration provides a complete, open-source solution
for deploying public or private ``Infrastructure as a Service'' (IaaS)
cloud infrastructures and is designed to be both simple to install and
simple to use.~\cite{slbook}

In parallel, the collaboration operates a federated cloud
infrastructure with two sites located in Orsay, France and Athens,
Greece\@.  These sites share a common user authentication framework
and appliance registry (the ``Marketplace'') allowing users to
allocate resources and to use appliances on either site.  The
StratusLab collaboration uses this production cloud to validate its
software in real world conditions.

\subsection{Marketplace}

In addition to the usual compute, storage, and networking services of
a IaaS cloud, StratusLab includes a complete set of tools for managing
appliances.  These tools cover:
\begin{itemize}
\item Creation of appliances in a secure and reproducible fashion,
\item Storage of appliances in an accessible location,
\item Efficient transport of appliances to the cloud,
  and
\item Metadata management for the appliances so that their contents
  are understood.
\end{itemize}
All cloud distributions provide tools to use appliances, but few
provide all of the tools necessary to share appliances between users
and between clouds.  The key feature for enabling appliance management
in federated cloud environments is the last one: one must have an open
registry that maintains appliance metadata.  The StratusLab
Marketplace fills this role.

\subsection{Organization}

This paper first discusses (in sections II and III) the features
necessary for sharing appliances between users and between cloud
infrastructures: appliance portability and tools for appliance
management.  It then describes the use cases, design, requirements,
implementation, and security considerations of the StratusLab
Marketplace (in sections IV--VII).  It then justifies the planned
improvements to the Marketplace through our experience in running it
in production for more than two years.
