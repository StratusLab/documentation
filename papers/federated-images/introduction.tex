\section{Introduction}
\label{sec:Introduction}

Cloud technologies are revolutionizing the way in which centralized
computing infrastructures are exposed to and used by scientists and
engineers.  These technologies enable rapid provisioning, dynamic
scaling, and environment customization that was impossible to achieve
previously. 

In particular, the ability to customize fully the execution
environment has drastically increased the utility of shared computing
infrastructures.  Users can create ``appliances'' that have their
desired operating system and required applications pre-installed
and pre-configured for their needs, independently of the needs of the
other users.

Being able to create, use, and reuse such appliances eliminates:
\begin{itemize}
\item The need for multiple installations of applications,
\item The costs associated with porting and re-validation of
  applications in varying envionments, and
\item Errors associated with running in different environments,
\end{itemize}
all together allowing users to concentrate on their calculations
rather than the software.  However, some of these reductions in effort
are offset by the need to manage the appliances themselves.  Tools for
this appliance management vary greatly between cloud distributions and
few are intended to be used in federated cloud environments.

\subsection{StratusLab}

The StratusLab collaboration provides a complete, open-source solution
for deploying public or private ``Infrastructure as a Service'' (IaaS)
cloud infrastructures and is designed to be both simple to install and
simple to use.~\cite{slbook} It provides the compute, storage, and
networking services expected in IaaS clouds.

The collaboration also operates a federated cloud infrastructure with
two sites--one in Orsay, France and the other in Athens, Greece\@.
These sites share a common user authentication framework and appliance
registry (the ``Marketplace'') allowing users to easily allocate
resources on either site.  This production cloud allows the StratusLab
collaboration to validate its software in real world conditions.

\subsection{Marketplace}

StratusLab provides a complete set of tools for managing appliances.
These tools cover:
\begin{itemize}
\item Creation of appliances in a secure and reproducible fashion,
\item Storage of appliances in an accessible location,
\item Efficient transport of appliances from the storage to the cloud,
  and
\item Metadata for the appliances so that their contents are
  understood. 
\end{itemize}
Although all cloud distributions provide tools to use appliances, few
provide all of the tools necessary to share appliances, especially in
federated cloud environments.  The key point in federated cloud
environments is the last one: one must have an open registry
that maintains appliance metadata.  This is the StratusLab
Marketplace\@. 

\subsection{Organization}

This paper first discusses (in sections II and III) the requirements
for sharing appliances between users and between cloud
infrastructures: appliance portability and tools for appliance
management.  It then describes the use cases, design, requirements,
implementation, and security considerations of the StratusLab
Marketplace (in sections IV--VII).  It then motivates future
improvements by discussing our experience running the Marketplace in
production for more than two years.

