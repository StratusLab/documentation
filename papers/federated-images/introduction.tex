\section{Introduction}
\label{sec:Introduction}

Rapid resource provisioning, dynamic scaling, and customized computing
environments make cloud infrastructures a compelling choice for a wide
range of scientific and engineering applications.  Federated cloud
infrastructures potentially offer further advantages such as access to
a larger number of resources and geographic redundancy for critical
applications.

Users, however, will only use federated cloud infrastructures if the
advantages outweigh the additional overheads.  To minimize these
overheads, much work has been done on standardizing the cloud
management interfaces (e.g. CIMI~\cite{cimi} and OCCI~\cite{occi}).
Unfortunately, the more important areas of appliance portability and
management that are required for a consistent computing environment
across cloud infrastructures have received much less attention.

StratusLab provides a complete, open-source solution for deploying
public or private ``Infrastructure as a Service'' (IaaS) cloud
infrastructures and is designed to be both simple to install and
simple to use~\cite{slbook}. It provides services similar to those in
the more widely known distributions like OpenStack~\cite{openstack}
and CloudStack~\cite{cloudstack}.

A distinguishing feature of StratusLab is its Marketplace, a
platform-agnostic appliance registry that facilitates sharing of
appliances and their use on multiple cloud infrastructures.  This
lowers barriers for potential cloud users in both federated and
non-federated cloud environments.

This paper first discusses (Sec.~\ref{sec:portable-appliances}) what
is required for creating portable appliances. It then highlights the
required appliance management features by describing the primary use
cases (Sec.~\ref{sec:use-cases}) and provides examples of how these
features are implemented in the StratusLab and other cloud
distributions (Sec.~\ref{sec:other-approaches}).  The core of the
paper (Sec.~\ref{sec:design}--\ref{sec:security}) describes the
Marketplace: its design, implementation, and associated security
concerns.  A discussion of our experience in running this service and
planned improvements based on that experience is then given
(Sec.~\ref{sec:production}). A description of a Marketplace 
deployed to support a particular user community (i.e., bioinformatics)
 concludes the paper (Sec.~\ref{sec:bioinfo}).
