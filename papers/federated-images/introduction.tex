\section{Introduction}
\label{sec:Introduction}

\subsection{Context}

{\em General description of the problem and the context.  This should
  describe why having a registry of some kind is required for IaaS
  clouds.  It should then explain the difficulties in creating good,
  secure images and how these can be reduced through sharing of
  existing images.  (Also make the point that although fully
  customized images are ideal, many people use similar images so there
  is a ``market'' for sharing.)  This section should also include a
  short ``publicity'' section about the StratusLab project and the
  scope of what is has accomplished.}

StratusLab provides a complete, open-source solution for deploying an
``Infrastructure as a Service'' cloud infrastructure.  Use of the
cloud requires the use of prepared machine and disk images.  Although
StratusLab provides tools to simplify the creation of these images,
the procedure for doing so remains a significant hurdle for use of a
cloud.  Consequently, StratusLab encourages the sharing and reuse of
existing images to reduce this barrier.

\subsection{Goal}

{\em Promote ecosystem of trusted applicances/images allowing users to
  more effectively and efficiently exploit IaaS cloud
  infrastructures.}

\subsection{Actors}

{\em Describe the actors involved in the ecosystem and their trust
  concerns.  The actors are: creators, endorsers (possibly separate
  from creators), users, administrators, and the Marketplace (as
  broker).  This should provide a framework in which to evaluate
  approaches in other cloud implementations and motivate the
  Marketplace requirements that appear later.}
