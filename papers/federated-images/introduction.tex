\section{Introduction}
\label{sec:Introduction}

{\em Describe the activities for the federation of cloud
  infrastructures.  Remark that most of this effort has concentrated
  on having a common cloud management API/interface.  The other
  aspects, especially shared, portable appliances have been
  neglected.}

{\em General introduction to image management in cloud environments.
  Point out the main features needed for image management and use:
  storage, transport, generation, and trust.  This last point is
  the crucial piece in a federated environment and what the StratusLab
  Marketplace provides.}

{\em Sharing of images requires portable images.  Being able to share
  images reduces the hurdle for using cloud infrastructures for new
  users.}

\subsection{StratusLab}

{\em Publicity for what StratusLab provides.  Statement that we've run
  a federated cloud infrastructure for more than two years.
  Concentrate on the Marketplace as an appliance registry.}

StratusLab provides a complete, open-source solution for deploying an
``Infrastructure as a Service'' cloud infrastructure.  Use of the
cloud requires the use of prepared machine and disk images.  Although
StratusLab provides tools to simplify the creation of these images,
the procedure for doing so remains a significant hurdle for use of a
cloud.  Consequently, StratusLab encourages the sharing and reuse of
existing images to reduce this barrier.

\subsection{Goal}

{\em Promote ecosystem of trusted applicances/images allowing users to
  more effectively and efficiently exploit IaaS cloud
  infrastructures.}

\subsection{Actors}

{\em Describe the actors involved in the ecosystem and their trust
  concerns.  The actors are: creators, endorsers (possibly separate
  from creators), users, administrators, and the Marketplace (as
  broker).  This should provide a framework in which to evaluate
  approaches in other cloud implementations and motivate the
  Marketplace requirements that appear later.}
