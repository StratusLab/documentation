\section{Introduction}
\label{sec:Introduction}

Rapid resource provisioning, dynamic scaling, and customized computing
environments make cloud infrastructures a compelling choice for a wide
range of scientific and engineering applications.  Federated cloud
infrastructures potentially offer further advantages such improved
scaling via access to a larger pool of resources and improved service
availability through geographically distributed redundant servers.

Users, however, will only use federated cloud infrastructures if the
advantages outweigh the additional overheads.  Existing and well
tested techniques for federated identity management, used in cluster,
grid, and commercial services, can provide unified access to the
federated clouds.  Specifically for clouds, much work has been done on
standardizing the cloud management interfaces (e.g. CIMI~\cite{cimi}
and OCCI~\cite{occi}).  Unfortunately, the critical areas of appliance
portability and management, required for consistent computing
environments across cloud infrastructures, have received much less
attention.

A distinguishing feature of the StratusLab~\cite{slbook} cloud
distribution is its Marketplace, a platform-agnostic appliance
registry that facilitates sharing of appliances and their use on
multiple cloud infrastructures.  This service and the associated
appliance management techniques lower barriers for users in both
federated and non-federated cloud environments.

StratusLab provides a complete, open-source solution for deploying
public or private ``Infrastructure as a Service'' (IaaS) cloud
infrastructures and is designed to be both simple to install and
simple to use.  In addition to the Marketplace, it provides services
similar to those in the more widely known distributions like
OpenStack~\cite{openstack} and CloudStack~\cite{cloudstack}.

This paper first discusses (Sec.~\ref{sec:portable-appliances}) what
is required for creating portable appliances. It then highlights the
required appliance management features by describing the primary use
cases (Sec.~\ref{sec:use-cases}) and provides examples of how these
features are implemented in the StratusLab and other cloud
distributions (Sec.~\ref{sec:other-approaches}).  The core of the
paper (Sec.~\ref{sec:design}--\ref{sec:security}) describes the
Marketplace: its design, implementation, and associated security
concerns.  A discussion of our experience in running this service and
planned improvements based on that experience is then given
(Sec.~\ref{sec:production}). A description of a Marketplace deployed
to support a particular user community (i.e., bioinformatics)
concludes the paper (Sec.~\ref{sec:bioinfo}).
