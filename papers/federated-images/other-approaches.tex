\section{Existing Appliance Management Techniques}
\label{sec:other-approaches}

In addition to the usual compute, storage, and networking services of
a IaaS cloud, StratusLab includes a complete set of tools for managing
appliances.  These tools cover:
\begin{itemize}
\item Creation of appliances in a secure and reproducible fashion,
\item Storage of appliances in an accessible location,
\item Efficient transport of appliances to the cloud,
  and
\item Metadata management for the appliances so that their contents
  are understood.
\end{itemize}
All cloud distributions provide tools to use appliances, but few
provide all of the tools necessary to share appliances between users
and between clouds.  The key feature for enabling appliance management
in federated cloud environments is the last one: one must have an open
registry that maintains appliance metadata.  The StratusLab
Marketplace fills this role.

\subsection{Appliance Repositories (storage)}

In OpenStack project ~\cite{openstack}, VM appliances can be stored in a variety of
backends from simple filesystems to object-storage systems like the
Swift project. The appliances are made available through a service
called Glance.

Glance, the OpenStack Appliance service, provides discovery,
registration and delivery for virtual machines appliances. It has a
RESTful API that allows querying of VM appliance metadata as well as
retrieval of the appliance itself.  Glance RESTful API exposes both
metadata about registered virtual machine appliances and the appliance
data itself.

In StratusLab, VM appliances can be stored in an appliance repository,
or in the storage service. Appliances stored in the appliance
repository are publics, and appliances stored in the persistent disk
storage are private.  An appliance repository is essentially a web
server that provides access to the contents of virtual machine
appliances. StratusLab does not provide a public appliance repository,
so currently custom appliances need to be uploaded somewhere where are
accessible by http. However, StratusLab provides standard base
appliances and partners appliances via a private repository appliance,
via which appliances can be shared. The appliances are available via
the http protocol, and are organizes by types and operating systems.
Current types are: grid for specialized grid appliances; base, for
standard OS appliances; bio for bioinformatics appliances; cernvm for
CERNVM appliances.

\subsection{Appliance Factories (generation)}

Generation of virtual machine appliances, could be done via different
cloud tools, like Bitnami ~\cite{bitnami}, puppet ~\cite{puppet}, StratusLab create appliance tools,
etc.

Bitnami provides Virtual machines appliances that can be deployed with
several components installed by default in these appliances (called
Bitnami Stacks), and other applications to Amazon EC2. The deployment
is done via the Bitnami Cloud Hosting interface. Bitnami offers
Virtual appliances as VMware machines. In order to run them, a cloud
infrastructure based on VMware ESX, vCenter and/or VMware Server
hypervisors is needed. These appliances can be downloaded and
customized, they can serve as base appliances for creating new
appliances with new software installed and new configuration. They can
also be converted to other format supported by other clouds.

Machine or disk appliances can either be created from scratch
following the StratusLab guidlines or be modified versions of existing
appliances.  StratusLab provides base minimal appliances for different
Linux distributions. These appliances could be used directly or
customized to create personalized machines.  The StratusLab
stratulab-create-appliance command tool permits to automate the
creation of a new appliance based on an existing one.
 
This tool takes three inputs:
\begin{itemize}
\item Marketplace identifier of the starting appliance,
\item A list of additional packages to install, and
\item A script to configuring the appliance.
\end{itemize}
stratus-create-appliance tool will install the package in the starting
appliance, execute the script user, and finally create the new
customized appliance with a new identifier.  The new generated
appliance will be stored in the storage service.

\subsection{Appliance Transport}

Appliances could be transferred to the cloud in differents way and
using different tools.

vmcaster/vmcatcher is a tool developed by HEPIX Virtualisation Working
Group ~\cite{hepixbooktransfer}. It has the property to download and transfer appliance from an
appliance list like do a Linux package manager, but rather for
appliances, authenticated by x509 signatures

vmcaster/vmcatcher has a concept based on the concept of VM appliance
list.  vmcatcher software makes use of database (essentially sqlite)
to store subscriptions to virtual machine appliances lists and permits
to download and transfer selected appliances from the list. The
downloaded appliances can be verified and cached.  The vmcaster tool
make appliance and appliance list update in simple way. It manage and
update published virtual machines appliance lists, store the old
appliance list, update an appliance and upload the signed appliance.

In StratusLab, appliances are downloaded and transferred from web
server or from storage. For an appliance to work, the location in the
appliance metadata must either be a reference to a persistent disk in
the storage service (for a private appliance) or an http(s) URL (for a
public appliance). Based on this information transfer of the appliance
is done in the cloud.  Download of the appliance is the only once. The
downloaded appliances are cached in the persistent disk storage.

\subsection{Appliance Registry}

The StratusLab Marketplace provides a registry for available, virtual
machine appliances. These are created by the people within the
community as well as by StratusLab partners to help people get started
using the cloud quickly.

Use of the cloud requires the use of prepared machine and disk
appliances. Although StratusLab provides tools to simplify the
creation of these appliances, the procedure for doing so remains a
significant hurdle for use of a cloud. Consequently, StratusLab
encourages the sharing and reuse of existing appliances to reduce this
barrier.  The Marketplace is at the center of the appliance handling
mechanisms in the StratusLab cloud distribution. It's a registry for
available, virtual machine appliances. These are created by the people
within the community as well as by StratusLab partners to help people
get started using the cloud quickly.  The Marketplace contains
metadata about appliances and serves as a registry for shared
appliances.

The metadata entry for an appliance is an XML file containing
information about the appliance itself including the location URLs for
the physical appliance. StratusLab provides simple tools to create
appliance metadata, singning and uploading it in the marketplace.  The
metadata is created using the StratusLab stratuslab-build-manifest
tool for building appliance metadata.  Signing Metadata means All of
the metadata entries must be signed with a valid p12 or grid
certificate. The stratus-sign-metadata tool will sign the metadata
file with a given certificate, inserting the certificate information
automatically into the metadata file.  The signed metadata entry can
then be uploaded to the StratusLab Marketplace. This is done via an
HTTP POST to the Marketplace.  The Marketplace will then validate the
entry and verify the email address given in the appliance
metadata. Once all of the checks have been done, the metadata will be
visible in the Marketplace and other users can search for the entry.

