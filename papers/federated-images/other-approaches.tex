\section{Existing Appliance Management Techniques}
\label{sec:other-approaches}

To manage appliances, cloud distributions must provide tools or
mechanisms for the:
\begin{itemize}
\item Creation of appliances,
\item Storage of appliances,
\item Efficient appliance transport, and
\item Management of appliance metadata.
\end{itemize}
This section describes the approaches used to implement these features
in StratusLab and other cloud distributions.

\subsection{Appliance Repositories (storage)}

Cloud distributions store VM appliances in a variety of ways. 
The OpenStack project~\cite{openstack} provides appliance discovery, registration and delivery, 
via its ``Glance'' service. The appliances can be stored in simple filesystems or 
object-storage systems like ``Swift''\@. Both metadata about registered appliances 
and the appliances themselves are exposed via the Glance API\@.

Eucalyptus~\cite{eucalyptus}, an open-source IaaS cloud distribution, provides an Amazon S3 interface to its 
``Walrus'' storage service. Virtual machine images are stored/retrieved using HTTP put/get.

OpenNebula~\cite{opennebula}, another IaaS distribution, uses the concept of a ``Datastore'' for storing appliances. 
Multiple datastores can be created backed by one of a selection of supported filesystem types. 
The method used to store and retrieve appliances depends on the type of datastore used (e.g. filesystem, 
iSCSI, Ceph). A set of ``transfer manager'' scripts that handle the interaction with the storage backend are provided for 
each of the types.

In StratusLab appliances can be stored in any web accessible (via HTTP(S)) location. The location of the appliance 
is contained in the metadata published in the Marketplace\@. This concept makes it possible 
to share appliances between StratusLab users, and also between users of different cloud 
infrastructures, because of the open accessibility and portability of StratusLab appliances (Sec.~\ref{sec:portable-appliances}). 

\subsection{Appliance Factories (generation)}

Several command line tools exist that automate the process of creating an appliance. In general these follow 
a similar pattern of installing a specified base operating system on a newly created disk image. The method used to 
customize the created appliance varies, and can be pre-defined or user generated templates, or through command line options. 
Examples of these types of tools include vmbuilder~\cite{vmbuilder}, VeeWee~\cite{veewee}, and BoxGrinder~\cite{boxgrinder}.

Bitnami is a service that provides pre-built, ready to deploy appliances.
These could be used as is, or treated as base appliances that a user can customize with their 
required software and configuration.

StratusLab provides ``image recipies'' for the automated creation of appliances. 
These recipies create minimal appliances for several Linux distributions. StratusLab uses 
the recipies to automatically update the standard base appliances provided. The StratusLab 
client also includes a tool that automates the creation of a new appliance from an existing base appliance.
The tool installs packages specified by the user and executes a user-defined  
script to configure the newly generated appliance. The new appliance is made available 
in the StratusLab storage service.

\subsection{Appliance Transport}

Appliances could be stored inside or outside of the cloud. In the latter case, 
appliances must be transported to the cloud for use.

The vmcaster/vmcatcher tools developed within the HEPIX Virtualisation Working
Group ~\cite{hepixbooktransfer}, use the concept of subscriptions to an 
appliance list. It makes the download and transport of an appliance 
from the appliance list similar to using a system package manager.
The downloaded appliances are
verified against their X509 signatures and cached.

In StratusLab, appliances are transported from a web
server or from cloud storage. Based on the appliance identifier in the Marketplace,
the transport of the appliance is done transparently by the cloud infrastructure.
The downloaded appliances are then verified and cached in the persistent disk storage,
ensuring that the transport of an appliance is done only once.

\subsection{Appliance Registry}

Examples of available appliance registeries include the Eucalyptus Image Store, the OpenNebula 
AppMarket and the StratusLab Marketplace.

The Eucalyptus Image Store provides a set of base appliances that can be downloaded and 
imported into a local Eucalyptus cloud using a command-line client.

The OpenNebula AppMarket allows registered appliance developers to upload appliances that 
users can then download to use in their local cloud. This is done using the OpenNebula command-line 
client, or through the ``Sunstone'' GUI.

The StratusLab Marketplace is at the center of the appliance handling
mechanisms in the StratusLab cloud distribution. It contains metadata 
about appliances and serves as a registry for shared appliances. 
In order to use and/or share an appliance, its metadata must be registered 
in the Marketplace\@. 

Once an appliance is created, StratusLab provides simple tools for 
building, cryptographically signing with a valid
certificate, and uploading the metadata to the Marketplace\@.  The Marketplace
validates the metadata entry and verifies the 
email address of the endorser.  If all the checks pass, 
the metadata will then be visible in the Marketplace and other users 
can search for the entry.
