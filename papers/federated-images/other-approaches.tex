\section{Existing Appliance Management Techniques}
\label{sec:other-approaches}

To manage appliances, cloud distributions must provide tools or
mechanisms for the:
\begin{itemize}
\item Creation of appliances,
\item Storage of appliances,
\item Efficient appliance transport, and
\item Management of appliance metadata.
\end{itemize}
This section describes the approaches used to implement these features
in StratusLab and other cloud distributions.

\subsection{Appliance Repositories (storage)}

Cloud distributions store VM appliances in a variety of ways. 
For example, the OpenStack project~\cite{openstack} 
via its ``Glance'' service, provides appliance discovery, registration and delivery.
The appliances can be stored in simple filesystems or 
object-storage systems like ``Swift''\@. Both metadata about registered appliances 
and the appliances itself are exposed via the Glance API\@.

In StratusLab, appliances are stored in a web server making them accessible via HTTP;
metadata about those appliances are published in the Marketplace\@. This concept makes it possible 
to share appliances between StratusLab users, and also between users of different cloud 
infrastructure, because of the open accessibility and portability of StratusLab appliances (Sec.~\ref{sec:portable-appliances}). 

\subsection{Appliance Factories (generation)}
New appliances can be created from scratch or by modifying existing appliances.
Generation of appliances, could be done via different
cloud tools, like Bitnami~\cite{bitnami}, puppet~\cite{puppet}, StratusLab create-image tool,
etc.

Bitnami does not provide tools for creating appliances from scratch,
instead it provides appliances with components installed by default, ready to deploy.
These appliances can be downloaded and customized, they can serve as 
base appliances for creating new appliances with users required software 
and configuration. 

StratusLab provides ``image recipes'' for creating, in automatic way, appliances from scratch. 
It also provides create-image tool for creating appliances from existing one's.

StratusLab ``image recipes'' serve to create base minimal appliances for different
Linux distributions. Those recipes are used automatically by StratusLab to create 
and update base minimal appliances. In the other hand, StratusLab create-image tool 
permits to automate the creation of a new appliance based on an existing one.
It install in the new generated appliances users required packages and run users 
scripts configuration.
The new generated appliance are stored in the storage service.

\subsection{Appliance Transport}

Appliances could be in the cloud or outside the cloud. In the latter case, 
appliances are transported to the cloud in differents way and using different tools.

vmcaster/vmcatcher developed by HEPIX Virtualisation Working
Group ~\cite{hepixbooktransfer}, has a concept of subscriptions to an 
appliance list. It makes the  download and transport of an appliance 
from the appliance list like do a Linux package manager, but rather for
appliances, authenticated by x509 signatures. The downloaded appliance are
verified and cached.

In StratusLab, appliances are transported from web
server or from storage. Based on the appliance identifier from the Marketplace,
the transport of the appliance is done transparently by the cloud infrastructure.
Then the downloaded appliances are verified and cached in the persistent disk storage,
making the transport of an appliance an operation done only once.

\subsection{Appliance Registry}
The StratusLab Marketplace is at the center of the appliance handling
mechanisms in the StratusLab cloud distribution. It contains metadata 
about appliances and serves as a registry for shared appliances. 
In order to use and/or share an appliance, its metadata must be registred 
in the Marketplace. 

Once an appliance is created, StratusLab provides simple tools for 
building its metatada, signing it with a valid p12 or grid
certificate, and uploading it to the Marketplace.  Then the Marketplace
apply the policy of validating the metadata entry and verify the 
email address given in the appliance metadata, and if all the checks pass, 
the metadata will be visible in the Marketplace and other users 
can search for the entry.

