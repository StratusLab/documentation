\section{Portable Appliances}
\label{sec:portable-appliances}

Before one can talk about sharing appliances, the appliances
themselves must be portable--generic enough to appeal to different
users and technically capable of running on different cloud platforms.
This has been identified as an important challenge in federated
clouds.~\cite{Petcu} Technically it involves the contextualization and
format of images.

\paragraph*{Appliance Format}
Nearly every hypervisor uses a different appliance format, but
fortunately tools exist to easily convert appliances between the
common formats.  However to reduce the burden on users, cloud
infrastructures should accept all formats and make the necessary
conversions automatically.  This allows users to manage one appliance
rather than one per format.  StratusLab, for example, automatically
converts images to the ``raw'' format it uses internally. 

\paragraph*{Appliance Contextualization} This allows an appliance to
discover its ``context'' and to automatically configure services to
work properly on a given infrastructure.  Common mechanisms provide a
disk or web server with context information to virtual machine
instances.  With additional work, portable appliances can be created
that detect and use multiple contextualization mechanisms; StratusLab
supports CloudInit~\cite{cloudinit}, HEPiX~\cite{hepixbookcontext},
and OpenNebula\@ contextualization.  Fortunately, CloudInit is
becoming a {\em de facto} standard and will simplify both appliance
creation and contextualization support by cloud distributions.

Although standard appliance contextualization and formats would easy
the creation of portable appliances, it is already possible to do so.
In fact, StratusLab provides portable, minimal appliances for most
popular Linux distributions.
