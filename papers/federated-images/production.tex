\section{Production Experiences}
\label{sec:production}

The StratusLab collaboration had operated a federated cloud
infrastructure with two sites for more than two years.  A shared
Marketplace allows users of this federated cloud to use the same
appliances on both sites.  This experience has allowed us to validate
the StratusLab appliance handling workflows and the Marketplace itself
in real world conditions.

The Marketplace has been improved over time based on feedback from
users.  However, there are some outstanding points that need to be
addressed in the future.

Having a central Marketplace instance allows users to easily find all
of the appliances from a single location.  Similarly, it allows image
creators to upload the metadata just once.  However, the Marketplace
is consulted everytime a new machine instance is launched to check if
an appliance has been deprecated.  Consequently if the Marketplace is
not available, new images cannot be started on any cloud relying on
the Marketplace\@.  Future iterations of the Marketplace must provide
redundancy and high-availability of the Marketplace service. 

By design the appliance metadata is considered public.  In reality,
however, there are both users and administrators that would like to
restrict the visibility of the appliance metadata.  Many cloud
administrators would like to run a ``private'' Marketplace to limit
the visibility of certain appliances while still taking advantage of
the central, public Marketplace instance.  There must be a mechanism
for federating Marketplace instances in the future. 

Although the metadata contains a significant amount of information
about an appliance, it does not contain information about how well the
appliance functions for users.  A common request has been to add
``social'' features to the Marketplace to allow users of an appliance
to leave comments and to signal problems with the appliance itself.
