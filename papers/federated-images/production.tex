\section{Production Experience \& Planned Improvements}
\label{sec:production}

In parallel with its software development, the collaboration operates
a federated cloud infrastructure with sites in Orsay, France and
Athens, Greece\@.  These sites share a common user authentication
framework and Marketplace allowing users to allocate resources and to
use appliances on either site.  The collaboration uses this production
cloud to validate its software in real world conditions.

\begin{table}
\caption{}
\label{tab:statistics}
\begin{center}
\begin{tabular}{ll}
\hline
number of endorsers & 1000 \\
number of appliances & 1000 \\
number of current appliances & 1000 \\
etc. & infinity \\
\hline
\end{tabular}
\end{center}
\end{table}

As a core service, the Marketplace is heavily used by both members of
the collaboration and users of the federated cloud.  Some statistics
on its use can be found in Table~\ref{tab:statistics}.  Overall the
service has performed well, with problems being addressed as the
software evolves over time.  Some outstanding points and potential
solutions are described below.

\subsection{Availability}

Having a central Marketplace instance allows users to easily find all
of the appliances from a single location.  Similarly, it allows image
creators to upload the metadata just once.  However, the Marketplace
is consulted everytime a new machine instance is launched to check if
an appliance has been deprecated.  Consequently if the Marketplace is
not available, new images cannot be started on any cloud relying on
the Marketplace\@.  Future iterations of the Marketplace must provide
redundancy and high-availability of the Marketplace service. 

As described above the Marketplace currently represents a single point
of failure in the StratusLab infrastructure. This is a critical issue
that is to be addressed in upcoming releases. A replication scheme
will be implemented to allow multiple Marketplace instances to be run,
each maintaining its own local copy of the global metadata
database. As all the information required to rebuild the metadata
database stored in Sesame is the set of raw metadata files, it is only
these that need be replicated. A potential solution to provide this
replication is to use a Git repository as the core `database' for the
metadata entries. This only requires that a Marketplace database be
updated periodically from a local clone of the global repository.

\subsection{Private Marketplaces}

An added benefit of this approach is that it would allow for the
public/private Marketplace requirment described in
Section~\ref{sec:production}. For a public Marketplace, the repository
could be hosted on a public service, such as GitHub~\cite{github} with
each replicated server pulling/pushing from that repository.  For a
private Marketplace, the repository can exist only on the local
disk. An alternative solution could be to use a distributed database,
such as CouchBase~\cite{couchbase}, to maintain the global
database. Updates of replica instances would be handled periodically
in much the same way as with Git.

\subsection{Data Protection}

By design the appliance metadata is considered public.  In reality,
however, there are both users and administrators that would like to
restrict the visibility of the appliance metadata.  Many cloud
administrators would like to run a ``private'' Marketplace to limit
the visibility of certain appliances while still taking advantage of
the central, public Marketplace instance.  There must be a mechanism
for federating Marketplace instances in the future. 

\subsection{Appliance Quality}

Although the metadata contains a significant amount of information
about an appliance, it does not contain information about how well the
appliance functions for users.  A common request has been to add
``social'' features to the Marketplace to allow users of an appliance
to leave comments and to signal problems with the appliance itself.

Improvements to the interface will include the ``social-features''
mentioned above, along with re-designing the various displays, such as
the appliance list and the metadata detailed view. As stated the
purpose of the social features would be to allow users to determine
the quality of an appliance referred to by a metadata entry. This
could be done with a comment, or rating, system. A simple way to add
this would be to make use of an external service such as
Disqus~\cite{disqus}.

\subsection{Stable URLs}

An important ability is to trace the evolution of an image through its
metadata. Currently modifying an appliance results in a new identifier
being created, which will then be used in the updated metadata. This
breaks the link with the previous version. The recently added `tag'
functionality goes someway to solving this.  The metadata entries for
a particular appliance can be found by retrieving all entries
containing a particular tag. This element is optional, however, and
may not be added to all entries. A more rigourous solution would be to
make use of the Dublin Core terms `replaces' and `isReplacedBy'. This
would provide a link in each metadata entry to the previous and next
entries in the lifetime of an appliance. Using these terms could add a
significant burden to the user creating a metadata entry, so this
would have to be automated as much as possible.
