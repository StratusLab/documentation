\section{Production Experience \& Planned Improvements}
\label{sec:production}

In parallel with its software development, the collaboration operates
a federated cloud infrastructure with sites in Orsay, France and
Athens, Greece\@.  These sites share a common user authentication
framework and Marketplace allowing users to allocate resources and to
use appliances on either site.  The collaboration uses this production
cloud to validate its software in real world conditions.

\begin{table}
\caption{}
\label{tab:statistics}
\begin{center}
\begin{tabular}{ll}
\hline
number of endorsers & 44 \\
number of current appliances & 105 \\
number of deprecated appliances & 170 \\
number of expired appliances & 832 \\
\hline
\end{tabular}
\end{center}
\end{table}

As a core service, the Marketplace is heavily used by both members of
the collaboration and users of the federated cloud.  Some statistics
on its use can be found in Table~\ref{tab:statistics}.  Overall the
service has performed well, with problems being addressed as the
software evolves over time.  Some outstanding points and potential
solutions are described below.

\subsection{Availability}

Having a central Marketplace instance allows users to easily find all
of the appliances from a single location.  Similarly, it allows image
creators to upload the metadata just once.  However, the Marketplace
is consulted everytime a new machine instance is launched to check if
an appliance has been deprecated.  Consequently if the Marketplace is
not available, new images cannot be started on any cloud relying on
the Marketplace\@.  Future iterations of the Marketplace must provide
redundancy and high-availability of the Marketplace service. 

To provide for this a replication scheme will be implemented that will 
allow multiple Marketplace instances to be run, each maintaining its own 
local copy of the global metadata database. As all the information required 
to rebuild the metadata database stored in Sesame is the set of raw metadata 
files, it is only these that need be replicated.  A potential solution to provide this
replication is to use a Git repository as the core `database' for the
metadata entries. This only requires that a Marketplace database be
updated periodically from a local clone of the global repository.

\subsection{Data Protection}

By design the appliance metadata is considered public.  In reality,
however, there are both users and administrators that would like to
restrict the visibility of the appliance metadata.  Many cloud
administrators would like to run a ``private'' Marketplace to limit
the visibility of certain appliances while still taking advantage of
the central, public Marketplace instance.  There must be a mechanism
for federating Marketplace instances in the future. 

This requirement could be integrated with the replication mechanism described above.  
For a public Marketplace, the repository could be hosted on a public service, 
such as GitHub~\cite{github} with each replicated server pulling/pushing 
from that repository.  For a private Marketplace, the repository can exist 
only on the local disk.

\subsection{Appliance Quality}

Although the metadata contains a significant amount of information
about an appliance, it does not contain information about how well the
appliance functions for users.  A common request has been to add
social features to the Marketplace to allow users of an appliance
to leave comments and to signal problems with the appliance itself.  
A possible approach to add these features without overly complicating 
the Marketplace implementation would be to make use of an external service 
such as Disqus~\cite{disqus}.

\subsection{Stable URLs}

An important ability is to trace the evolution of an image through its
metadata. Currently modifying an appliance results in a new identifier
being created, which will then be used in the updated metadata. This
breaks the link with the previous version. The recently added `tag'
functionality goes someway to solving this.  The metadata entries for
a particular appliance can be found by retrieving all entries
containing a particular tag. This element is optional, however, and
may not be added to all entries. A more rigourous solution would be to
make use of the Dublin Core terms `replaces' and `isReplacedBy'. This
would provide a link in each metadata entry to the previous and next
entries in the lifetime of an appliance. Using these terms could add make it 
more difficult for a user to create a metadata entry, and so should be 
automated as much as possible.
