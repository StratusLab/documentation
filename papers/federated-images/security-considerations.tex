\section{Security Considerations}
\label{sec:security}

\subsection{Security Policies}

{\em Various scenarios that have been identified and how the system
  works to prevent security incidents.}

{\em Describe efforts to work towards general policies in the EGI
  community concerning the handling and trust of virtual machine
  images.  Also efforts towards wider standardization, if
  appropriate.}

\subsection{Policy Implementation}

{\em Describe the policy management tools and how they support
  flexible but rigorous site policies.}

\subsection{Specific Attacks}

\subsubsection{Replay of Signed Metadata}

As all of the signed metadata data entries are publicly available,
there is a risk that an older entry would be ``replayed'' to replace a
newer entry.  This would be done, for instance, to make an image that
has been deprecated to appear to still be valid.

The implementation makes this difficult to do by:
\begin{itemize}
\item Requiring that all metadata entries explicitly include the
  endorsement timestamp as part of the signed content.
\item Only allowing uploaded entries that have a timestamp that is 
  more recent than the current entry for the image.
\item Confirming all changes via the email address in the metadata
  entry.
\end{itemize}
To successfully replay a previous entry would require that the
endorser's private key be compromised (to update timestamp) as well as
his email account.

\subsubsection{Confidentiality of Data}

{\em Expand to copyright and IPR.  Safe haven provisions in not
  directly hosting images.}

As all of the metadata descriptions on the servers are considered
public, there are no concerns about confidentiality.  Thus the
Marketplace implementation does not need to do anything special.

\subsubsection{Validity and Completeness of Data}

{\em Probably split this into two sections.  Need to introduce the
  idea of a ``timeline'' of endorsements for an image.}

If there were an open communication channel between the Marketplace
and the user, then responses from the server could be altered in
transit.  To prevent this, the server must be deployed with a valid
certificate and communications must take place over a connection using
TLS\@.

\subsubsection{Altered Images}

As the images will be downloaded from other sites, there is a danger
that they will be altered (either intentionally or maliciously).  As
images are selected and evaluated based on the associated metadata,
altered images must not correspond to the unaltered image's metadata
or to another valid image's metadata.

The image identifier is based on the SHA-1 hash of the image.
Although modifying an image while maintaining the SHA-1 hash
is difficult, it is possible.  Hence an altered image could masquerade
as a valid image if only the SHA-1 hash information were used to
validate a downloaded image file.

To ensure that altered images are detected in the validation
additional information is provided in the metadata descriptions:
\begin{itemize}
\item Length of the file in bytes.
\item MD5, SHA-1, SHA-256, and SHA-512 hash values.
\end{itemize}
All of these should be verified with a newly downloaded image.  The
likelihood that someone can create an altered image with exactly the
same length and multiple checksums is miniscule.

\subsubsection{Compromised Marketplace Server}

If someone were to take control of the Marketplace server, he could
not alter individual metadata entries as those are signed by the
endorser's whose keys are not available on the server.  However, he
could delete entries making, for instance, deprecated images appear
valid.  This is a significant risk and the server should be operated
according to modern best practices.
