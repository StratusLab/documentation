\section{Security Considerations}
\label{sec:security}

The Marketplace and the information contained within the Marketplace
play a key role in maintaining the security of the cloud
infrastructures.  However the security policies both for the users and
for the cloud administrators can vary widely.  {\em Consequently, the
  Marketplace itself does not define or enforce any security policy.}
It instead securely provides the appliance metadata allowing both
users and cloud administrators to make informed decisions about the
appliances.

%The StratusLab distribution provides a python-based module that allows
%administrators to configure and to enforce several common policies,
%such an image or endorser white- and blacklists.  This module is
%easily extensible to allow more complicated policies where, for
%example, the certificates associated with the cryptographic signatures
%of endorsements are validated.  This module could easily be used
%within other cloud distributions.

To maintain confidence in the information provided by the Marketplace,
it must securely provide complete, accurate information about the
appliances.  We have identified a number of attack scenarios and
describe how the Marketplace has been designed to avoid them. 

\subsection{Verification of Uploaded Metadata}

When new metadata entries are uploaded to the Marketplace, the service
validates the entry before accepting it.  This involves three
separate steps:
\begin{enumerate}
\item The metadata must be a valid RDF/XML file, following the defined
  schemas and conventions.
\item To avoid having old metadata entries being reposted (for example
  to make a deprecated entry appear to be valid), the server
  ensures that only entries endorsed after the current entry for a particular
  appliance are accepted.
\item To ensure that the included email address is valid, all
  additions to the Marketplace must be confirmed by email.
\end{enumerate}
Only validated metadata entries are visible from the standard
Marketplace interface. 

\subsection{Replay of Signed Metadata}

As all of the signed metadata data entries are publicly available,
there is a risk that an older entry would be ``replayed'' to replace a
newer entry.  This would be done, for instance, to make an image that
has been deprecated to appear to still be valid.

The implementation makes this difficult to do by:
\begin{itemize}
\item Requiring that all metadata entries explicitly include the
  endorsement timestamp as part of the signed content.
\item Only allowing uploaded entries that have a timestamp that is 
  more recent than the current entry for the image.
\item Confirming all changes via the email address in the metadata
  entry.
\end{itemize}
To successfully replay a previous entry would require that the
endorser's private key be compromised (to update timestamp) as well as
his email account.

\subsection{Confidentiality of Data}

As all of the metadata descriptions on the servers are considered
public.  Consequently, there are no concerns about confidentiality of
the metadata and the Marketplace implementation does not need to do
anything special.

However, the images themselves contain software and data that are
subject to copyright and possibly licensing restrictions.  The
StratusLab design, where the storage of an appliance is separate from
its metadata, avoids having to deal with intellectual property rights
violations.

\subsection{Validity and Completeness of Data}

The Marketplace must ensure that all of the data concerning an
appliance is available.  By design, the Marketplace never removes
metadata entries--the entire appliance timeline is always available.
By default however, only the current endorsements are provided as
these are the entries needed to make decisions about the validity of
an image.  

The Marketplace must ensure that the data transmitted to users is not
altered, for example by a third-party removing deprecation notices
from the returned information.  To prevent this, the Marketplace only
transmits information over a communication channel secured via TLS\@. 

\subsection{Altered Appliances}

As the appliances will be downloaded from other sites, there is a danger
that they will be altered (either intentionally or maliciously).  As
appliances are selected and evaluated based on the associated
metadata, the association between the metadata and the image must be
absolute. 

The appliance identifier is based on the SHA-1 hash of the image.
Although modifying an image while maintaining the SHA-1 hash is
difficult, it is a remote possibility.  Hence an altered appliance
could masquerade as a valid appliance if only the SHA-1 hash
information were used to validate a downloaded image file.

To minimize the possibility that an altered appliance appears to be
valid, additional information is provided in the metadata
descriptions: the size of the file in bytes and the MD5, SHA-1,
SHA-256, and SHA-512 hash values.  All of these can and should be
verified with a newly downloaded image.  The likelihood that someone
can create an altered appliance with exactly the same length and
multiple checksums is miniscule.

\subsection{Compromised Marketplace Server}

If someone were to take control of the Marketplace server, he could
not alter individual metadata entries as those are signed by the
endorsers' private keys are not stored on the server.  However, he
could delete entries making, for instance, deprecated images appear
valid.  This is a significant risk and the server should be operated
according to modern best practices to avoid this.
