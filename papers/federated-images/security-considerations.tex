\section{Security Considerations}
\label{sec:security}

The Marketplace and the information contained within the Marketplace
play a key role in maintaining the security of the cloud
infrastructures.  However, the security policies both for the users
and for the cloud administrators can vary widely and {\em
  Consequently, the Marketplace itself does not define or enforce any
  security policy.}  It instead provides the appliance metadata
allowing both users and cloud administrators to make informed
decisions about the appliances.

To maintain confidence in the information provided by the Marketplace,
it must securely provide complete, accurate information about the
appliances.  We have identified a number of security concerns and
describe how the Marketplace solves them. 

\subsection{Altered Appliances}

Because the appliance metadata is separated from the appliance
contents, there is a danger that the appliance contents could be
altered, either maliciously or unintentionally.  As described above,
the appliance identifier is based on the SHA-1 hash of the image
ensuring a very reliable connection between the metadata and the
appliance contents.

Although modifying an image while maintaining the SHA-1 hash is
difficult, it is a remote possibility, allowing the altered appliance
to masquerade as the original.  To minimize this possibility,
additional information is provided in the metadata descriptions: the
size of the file in bytes and the MD5, SHA-1, SHA-256, and SHA-512
hash values.  The likelihood that someone can create an altered
appliance with exactly the same length and multiple checksums is
miniscule.

\subsection{Verification of Uploaded Metadata}

When new metadata entries are uploaded to the Marketplace, the service
validates the entry before accepting it.  To validate the entries, the server:
\begin{itemize}
\item Verifies that the metadata is a valid RDF/XML file, following
  the defined schemas and conventions,
\item Accepts only entries endorsed after the most recent entry for a
  particular appliance are accepted, and 
\item Confirms the email address of the entry.
\end{itemize}
Only validated metadata entries are visible from the standard
Marketplace interface.

\subsection{Altered Appliance Timeline}

The Marketplace must ensure that all of the data concerning an
appliance is available.  By design, the Marketplace never removes
metadata entries--the entire appliance timeline is always available.
By default however, only the current endorsements are provided as
these are the entries needed to make decisions about the validity of
an image.  

The Marketplace must ensure that the data transmitted to users is not
altered, for example by a third-party removing deprecation notices
from the returned information.  To prevent this, the Marketplace only
transmits information over a communication channel secured via TLS\@.

In addition, it does not allow the history of an appliance to be
altered.  As described above, the Marketplace does not accept entries
with a timestamp earlier than the latest entry in the timeline.  It
also validates the endorser to avoid one endorser from impersonating
another.

\subsection{Compromised Marketplace Server}

If someone were to take control of the Marketplace server, he could
not alter individual metadata entries as those are signed by the
endorsers' private keys are not stored on the server.  However, he
could delete entries making, for instance, deprecated images appear
valid.

This is a significant risk and the server must be operated according
to modern best practices to avoid this.  In addition, backups of the
information should be kept and periodically compared to the current
information to detect any such attack.
