
%% bare_conf.tex
%% V1.3
%% 2007/01/11
%% by Michael Shell
%% See:
%% http://www.michaelshell.org/
%% for current contact information.
%%
%% This is a skeleton file demonstrating the use of IEEEtran.cls
%% (requires IEEEtran.cls version 1.7 or later) with an IEEE conference paper.
%%
%% Support sites:
%% http://www.michaelshell.org/tex/ieeetran/
%% http://www.ctan.org/tex-archive/macros/latex/contrib/IEEEtran/
%% and
%% http://www.ieee.org/

%%*************************************************************************
%% Legal Notice:
%% This code is offered as-is without any warranty either expressed or
%% implied; without even the implied warranty of MERCHANTABILITY or
%% FITNESS FOR A PARTICULAR PURPOSE! 
%% User assumes all risk.
%% In no event shall IEEE or any contributor to this code be liable for
%% any damages or losses, including, but not limited to, incidental,
%% consequential, or any other damages, resulting from the use or misuse
%% of any information contained here.
%%
%% All comments are the opinions of their respective authors and are not
%% necessarily endorsed by the IEEE.
%%
%% This work is distributed under the LaTeX Project Public License (LPPL)
%% ( http://www.latex-project.org/ ) version 1.3, and may be freely used,
%% distributed and modified. A copy of the LPPL, version 1.3, is included
%% in the base LaTeX documentation of all distributions of LaTeX released
%% 2003/12/01 or later.
%% Retain all contribution notices and credits.
%% ** Modified files should be clearly indicated as such, including  **
%% ** renaming them and changing author support contact information. **
%%
%% File list of work: IEEEtran.cls, IEEEtran_HOWTO.pdf, bare_adv.tex,
%%                    bare_conf.tex, bare_jrnl.tex, bare_jrnl_compsoc.tex
%%*************************************************************************

% *** Authors should verify (and, if needed, correct) their LaTeX system  ***
% *** with the testflow diagnostic prior to trusting their LaTeX platform ***
% *** with production work. IEEE's font choices can trigger bugs that do  ***
% *** not appear when using other class files.                            ***
% The testflow support page is at:
% http://www.michaelshell.org/tex/testflow/



% Note that the a4paper option is mainly intended so that authors in
% countries using A4 can easily print to A4 and see how their papers will
% look in print - the typesetting of the document will not typically be
% affected with changes in paper size (but the bottom and side margins will).
% Use the testflow package mentioned above to verify correct handling of
% both paper sizes by the user's LaTeX system.
%
% Also note that the "draftcls" or "draftclsnofoot", not "draft", option
% should be used if it is desired that the figures are to be displayed in
% draft mode.
%
\documentclass[conference]{IEEEtran}
% Add the compsoc option for Computer Society conferences.
%
% If IEEEtran.cls has not been installed into the LaTeX system files,
% manually specify the path to it like:
% \documentclass[conference]{../sty/IEEEtran}


\newcommand{\replaceable}[1]{\ensuremath{\langle}#1\ensuremath{\rangle}}
\newcommand{\cmd}[1]{\texttt{#1}}
\newcommand\tnl{\tabularnewline[1mm]}

% Some very useful LaTeX packages include:
% (uncomment the ones you want to load)


% *** MISC UTILITY PACKAGES ***
%
%\usepackage{ifpdf}
% Heiko Oberdiek's ifpdf.sty is very useful if you need conditional
% compilation based on whether the output is pdf or dvi.
% usage:
% \ifpdf
%   % pdf code
% \else
%   % dvi code
% \fi
% The latest version of ifpdf.sty can be obtained from:
% http://www.ctan.org/tex-archive/macros/latex/contrib/oberdiek/
% Also, note that IEEEtran.cls V1.7 and later provides a builtin
% \ifCLASSINFOpdf conditional that works the same way.
% When switching from latex to pdflatex and vice-versa, the compiler may
% have to be run twice to clear warning/error messages.






% *** CITATION PACKAGES ***
%
%\usepackage{cite}
% cite.sty was written by Donald Arseneau
% V1.6 and later of IEEEtran pre-defines the format of the cite.sty package
% \cite{} output to follow that of IEEE. Loading the cite package will
% result in citation numbers being automatically sorted and properly
% "compressed/ranged". e.g., [1], [9], [2], [7], [5], [6] without using
% cite.sty will become [1], [2], [5]--[7], [9] using cite.sty. cite.sty's
% \cite will automatically add leading space, if needed. Use cite.sty's
% noadjust option (cite.sty V3.8 and later) if you want to turn this off.
% cite.sty is already installed on most LaTeX systems. Be sure and use
% version 4.0 (2003-05-27) and later if using hyperref.sty. cite.sty does
% not currently provide for hyperlinked citations.
% The latest version can be obtained at:
% http://www.ctan.org/tex-archive/macros/latex/contrib/cite/
% The documentation is contained in the cite.sty file itself.






% *** GRAPHICS RELATED PACKAGES ***
%
\ifCLASSINFOpdf
  % \usepackage[pdftex]{graphicx}
  % declare the path(s) where your graphic files are
  % \graphicspath{{../pdf/}{../jpeg/}}
  % and their extensions so you won't have to specify these with
  % every instance of \includegraphics
  % \DeclareGraphicsExtensions{.pdf,.jpeg,.png}
\else
  % or other class option (dvipsone, dvipdf, if not using dvips). graphicx
  % will default to the driver specified in the system graphics.cfg if no
  % driver is specified.
  % \usepackage[dvips]{graphicx}
  % declare the path(s) where your graphic files are
  % \graphicspath{{../eps/}}
  % and their extensions so you won't have to specify these with
  % every instance of \includegraphics
  % \DeclareGraphicsExtensions{.eps}
\fi
% graphicx was written by David Carlisle and Sebastian Rahtz. It is
% required if you want graphics, photos, etc. graphicx.sty is already
% installed on most LaTeX systems. The latest version and documentation can
% be obtained at: 
% http://www.ctan.org/tex-archive/macros/latex/required/graphics/
% Another good source of documentation is "Using Imported Graphics in
% LaTeX2e" by Keith Reckdahl which can be found as epslatex.ps or
% epslatex.pdf at: http://www.ctan.org/tex-archive/info/
%
% latex, and pdflatex in dvi mode, support graphics in encapsulated
% postscript (.eps) format. pdflatex in pdf mode supports graphics
% in .pdf, .jpeg, .png and .mps (metapost) formats. Users should ensure
% that all non-photo figures use a vector format (.eps, .pdf, .mps) and
% not a bitmapped formats (.jpeg, .png). IEEE frowns on bitmapped formats
% which can result in "jaggedy"/blurry rendering of lines and letters as
% well as large increases in file sizes.
%
% You can find documentation about the pdfTeX application at:
% http://www.tug.org/applications/pdftex





% *** MATH PACKAGES ***
%
%\usepackage[cmex10]{amsmath}
% A popular package from the American Mathematical Society that provides
% many useful and powerful commands for dealing with mathematics. If using
% it, be sure to load this package with the cmex10 option to ensure that
% only type 1 fonts will utilized at all point sizes. Without this option,
% it is possible that some math symbols, particularly those within
% footnotes, will be rendered in bitmap form which will result in a
% document that can not be IEEE Xplore compliant!
%
% Also, note that the amsmath package sets \interdisplaylinepenalty to 10000
% thus preventing page breaks from occurring within multiline equations. Use:
%\interdisplaylinepenalty=2500
% after loading amsmath to restore such page breaks as IEEEtran.cls normally
% does. amsmath.sty is already installed on most LaTeX systems. The latest
% version and documentation can be obtained at:
% http://www.ctan.org/tex-archive/macros/latex/required/amslatex/math/





% *** SPECIALIZED LIST PACKAGES ***
%
%\usepackage{algorithmic}
% algorithmic.sty was written by Peter Williams and Rogerio Brito.
% This package provides an algorithmic environment fo describing algorithms.
% You can use the algorithmic environment in-text or within a figure
% environment to provide for a floating algorithm. Do NOT use the algorithm
% floating environment provided by algorithm.sty (by the same authors) or
% algorithm2e.sty (by Christophe Fiorio) as IEEE does not use dedicated
% algorithm float types and packages that provide these will not provide
% correct IEEE style captions. The latest version and documentation of
% algorithmic.sty can be obtained at:
% http://www.ctan.org/tex-archive/macros/latex/contrib/algorithms/
% There is also a support site at:
% http://algorithms.berlios.de/index.html
% Also of interest may be the (relatively newer and more customizable)
% algorithmicx.sty package by Szasz Janos:
% http://www.ctan.org/tex-archive/macros/latex/contrib/algorithmicx/




% *** ALIGNMENT PACKAGES ***
%
%\usepackage{array}
% Frank Mittelbach's and David Carlisle's array.sty patches and improves
% the standard LaTeX2e array and tabular environments to provide better
% appearance and additional user controls. As the default LaTeX2e table
% generation code is lacking to the point of almost being broken with
% respect to the quality of the end results, all users are strongly
% advised to use an enhanced (at the very least that provided by array.sty)
% set of table tools. array.sty is already installed on most systems. The
% latest version and documentation can be obtained at:
% http://www.ctan.org/tex-archive/macros/latex/required/tools/


%\usepackage{mdwmath}
%\usepackage{mdwtab}
% Also highly recommended is Mark Wooding's extremely powerful MDW tools,
% especially mdwmath.sty and mdwtab.sty which are used to format equations
% and tables, respectively. The MDWtools set is already installed on most
% LaTeX systems. The lastest version and documentation is available at:
% http://www.ctan.org/tex-archive/macros/latex/contrib/mdwtools/


% IEEEtran contains the IEEEeqnarray family of commands that can be used to
% generate multiline equations as well as matrices, tables, etc., of high
% quality.


%\usepackage{eqparbox}
% Also of notable interest is Scott Pakin's eqparbox package for creating
% (automatically sized) equal width boxes - aka "natural width parboxes".
% Available at:
% http://www.ctan.org/tex-archive/macros/latex/contrib/eqparbox/





% *** SUBFIGURE PACKAGES ***
%\usepackage[tight,footnotesize]{subfigure}
% subfigure.sty was written by Steven Douglas Cochran. This package makes it
% easy to put subfigures in your figures. e.g., "Figure 1a and 1b". For IEEE
% work, it is a good idea to load it with the tight package option to reduce
% the amount of white space around the subfigures. subfigure.sty is already
% installed on most LaTeX systems. The latest version and documentation can
% be obtained at:
% http://www.ctan.org/tex-archive/obsolete/macros/latex/contrib/subfigure/
% subfigure.sty has been superceeded by subfig.sty.



%\usepackage[caption=false]{caption}
%\usepackage[font=footnotesize]{subfig}
% subfig.sty, also written by Steven Douglas Cochran, is the modern
% replacement for subfigure.sty. However, subfig.sty requires and
% automatically loads Axel Sommerfeldt's caption.sty which will override
% IEEEtran.cls handling of captions and this will result in nonIEEE style
% figure/table captions. To prevent this problem, be sure and preload
% caption.sty with its "caption=false" package option. This is will preserve
% IEEEtran.cls handing of captions. Version 1.3 (2005/06/28) and later 
% (recommended due to many improvements over 1.2) of subfig.sty supports
% the caption=false option directly:
%\usepackage[caption=false,font=footnotesize]{subfig}
%
% The latest version and documentation can be obtained at:
% http://www.ctan.org/tex-archive/macros/latex/contrib/subfig/
% The latest version and documentation of caption.sty can be obtained at:
% http://www.ctan.org/tex-archive/macros/latex/contrib/caption/




% *** FLOAT PACKAGES ***
%
%\usepackage{fixltx2e}
% fixltx2e, the successor to the earlier fix2col.sty, was written by
% Frank Mittelbach and David Carlisle. This package corrects a few problems
% in the LaTeX2e kernel, the most notable of which is that in current
% LaTeX2e releases, the ordering of single and double column floats is not
% guaranteed to be preserved. Thus, an unpatched LaTeX2e can allow a
% single column figure to be placed prior to an earlier double column
% figure. The latest version and documentation can be found at:
% http://www.ctan.org/tex-archive/macros/latex/base/



%\usepackage{stfloats}
% stfloats.sty was written by Sigitas Tolusis. This package gives LaTeX2e
% the ability to do double column floats at the bottom of the page as well
% as the top. (e.g., "\begin{figure*}[!b]" is not normally possible in
% LaTeX2e). It also provides a command:
%\fnbelowfloat
% to enable the placement of footnotes below bottom floats (the standard
% LaTeX2e kernel puts them above bottom floats). This is an invasive package
% which rewrites many portions of the LaTeX2e float routines. It may not work
% with other packages that modify the LaTeX2e float routines. The latest
% version and documentation can be obtained at:
% http://www.ctan.org/tex-archive/macros/latex/contrib/sttools/
% Documentation is contained in the stfloats.sty comments as well as in the
% presfull.pdf file. Do not use the stfloats baselinefloat ability as IEEE
% does not allow \baselineskip to stretch. Authors submitting work to the
% IEEE should note that IEEE rarely uses double column equations and
% that authors should try to avoid such use. Do not be tempted to use the
% cuted.sty or midfloat.sty packages (also by Sigitas Tolusis) as IEEE does
% not format its papers in such ways.





% *** PDF, URL AND HYPERLINK PACKAGES ***
%
\usepackage{url}
% url.sty was written by Donald Arseneau. It provides better support for
% handling and breaking URLs. url.sty is already installed on most LaTeX
% systems. The latest version can be obtained at:
% http://www.ctan.org/tex-archive/macros/latex/contrib/misc/
% Read the url.sty source comments for usage information. Basically,
% \url{my_url_here}.






% *** Do not adjust lengths that control margins, column widths, etc. ***
% *** Do not use packages that alter fonts (such as pslatex).         ***
% There should be no need to do such things with IEEEtran.cls V1.6 and later.
% (Unless specifically asked to do so by the journal or conference you plan
% to submit to, of course. )


% correct bad hyphenation here
\hyphenation{op-tical net-works semi-conduc-tor}


\begin{document}
%
% paper title
% can use linebreaks \\ within to get better formatting as desired
\title{Image Management in a Federated Cloud Environment}

% author names and affiliations
% use a multiple column layout for up to three different
% affiliations
\author{
\IEEEauthorblockN{Mohammed Airaj}
\IEEEauthorblockA{Linear Accelerator Laboratory (LAL)\\
Orsay, France and \\
Caddi Ayyad University\\
Marrakech, Morocco\\
Email: airaj@lal.in2p3.fr}
\and
\IEEEauthorblockN{Marc-Elian B\'{e}gin}
\IEEEauthorblockA{SixSq S\`{a}rl\\
Geneva, Switzerland\\
Email: meb@sixsq.com}%
\and
\IEEEauthorblockN{Evangelos Floros}
\IEEEauthorblockA{GRNET\\
Athens, Greece\\
Email: efloros@grnet.gr}%
\and
\IEEEauthorblockN{Stuart Kenny}
\IEEEauthorblockA{Trinity College Dublin\\
Dublin, Ireland\\
Email: stuart.kenny@cs.tcd.ie}
\and
\IEEEauthorblockN{Charles Loomis}
\IEEEauthorblockA{Linear Accelerator Laboratory (LAL)\\
Orsay, France\\
Email: loomis@lal.in2p3.fr}
\and
\IEEEauthorblockN{David O'Callaghan}
\IEEEauthorblockA{Trinity College Dublin\\
Dublin, Ireland\\
Email: david.ocallaghan@cs.tcd.ie}%
}

% conference papers do not typically use \thanks and this command
% is locked out in conference mode. If really needed, such as for
% the acknowledgment of grants, issue a \IEEEoverridecommandlockouts
% after \documentclass

% for over three affiliations, or if they all won't fit within the width
% of the page, use this alternative format:
% 
%\author{\IEEEauthorblockN{Michael Shell\IEEEauthorrefmark{1},
%Homer Simpson\IEEEauthorrefmark{2},
%James Kirk\IEEEauthorrefmark{3}, 
%Montgomery Scott\IEEEauthorrefmark{3} and
%Eldon Tyrell\IEEEauthorrefmark{4}}
%\IEEEauthorblockA{\IEEEauthorrefmark{1}School of Electrical and Computer Engineering\\
%Georgia Institute of Technology,
%Atlanta, Georgia 30332--0250\\ Email: see http://www.michaelshell.org/contact.html}
%\IEEEauthorblockA{\IEEEauthorrefmark{2}Twentieth Century Fox, Springfield, USA\\
%Email: homer@thesimpsons.com}
%\IEEEauthorblockA{\IEEEauthorrefmark{3}Starfleet Academy, San Francisco, California 96678-2391\\
%Telephone: (800) 555--1212, Fax: (888) 555--1212}
%\IEEEauthorblockA{\IEEEauthorrefmark{4}Tyrell Inc., 123 Replicant Street, Los Angeles, California 90210--4321}}




% use for special paper notices
%\IEEEspecialpapernotice{(Invited Paper)}




% make the title area
\maketitle

\begin{abstract}
Cloud infrastructures provide compelling features for scientific and
engineering applications.  Federated clouds additionally promise
geographic service redundancy and access to more resources.  Effective
use of federated clouds requires the creation of portable appliances
and consistent appliance management techniques.  The StratusLab
Marketplace, a platform-agnostic appliance registry, facilitates
appliance management in a federated environment.  This paper describes
the Marketplace design goals, implementation, and security concerns.
It also covers the planned improvements based on our experience of
running this service in production for more than two years.
\end{abstract}


% IEEEtran.cls defaults to using nonbold math in the Abstract.
% This preserves the distinction between vectors and scalars. However,
% if the conference you are submitting to favors bold math in the abstract,
% then you can use LaTeX's standard command \boldmath at the very start
% of the abstract to achieve this. Many IEEE journals/conferences frown on
% math in the abstract anyway.

% no keywords


% For peer review papers, you can put extra information on the cover
% page as needed:
% \ifCLASSOPTIONpeerreview
% \begin{center} \bfseries EDICS Category: 3-BBND \end{center}
% \fi
%
% For peerreview papers, this IEEEtran command inserts a page break and
% creates the second title. It will be ignored for other modes.
\IEEEpeerreviewmaketitle


\section{Introduction}
\label{sec:Introduction}

{\em Describe the activities for the federation of cloud
  infrastructures.  Remark that most of this effort has concentrated
  on having a common cloud management API/interface.  The other
  aspects, especially shared, portable appliances have been
  neglected.}

{\em General introduction to image management in cloud environments.
  Point out the main features needed for image management and use:
  storage, transport, generation, and trust.  This last point is
  the crucial piece in a federated environment and what the StratusLab
  Marketplace provides.}

{\em Sharing of images requires portable images.  Being able to share
  images reduces the hurdle for using cloud infrastructures for new
  users.}

\subsection{StratusLab}

{\em Publicity for what StratusLab provides.  Statement that we've run
  a federated cloud infrastructure for more than two years.
  Concentrate on the Marketplace as an appliance registry.}

StratusLab provides a complete, open-source solution for deploying an
``Infrastructure as a Service'' cloud infrastructure.  Use of the
cloud requires the use of prepared machine and disk images.  Although
StratusLab provides tools to simplify the creation of these images,
the procedure for doing so remains a significant hurdle for use of a
cloud.  Consequently, StratusLab encourages the sharing and reuse of
existing images to reduce this barrier.

\subsection{Goal}

{\em Promote ecosystem of trusted applicances/images allowing users to
  more effectively and efficiently exploit IaaS cloud
  infrastructures.}

\subsection{Actors}

{\em Describe the actors involved in the ecosystem and their trust
  concerns.  The actors are: creators, endorsers (possibly separate
  from creators), users, administrators, and the Marketplace (as
  broker).  This should provide a framework in which to evaluate
  approaches in other cloud implementations and motivate the
  Marketplace requirements that appear later.}


\section{Existing Appliance Management Techniques}
\label{sec:other-approaches}

To manage appliances, cloud distributions must provide tools or
mechanisms for the:
\begin{itemize}
\item Creation of appliances,
\item Storage of appliances,
\item Efficient appliance transport, and
\item Management of appliance metadata.
\end{itemize}
This section describes the approaches used to implement these features
in StratusLab and other cloud distributions.

\subsection{Appliance Repositories (storage)}

Cloud distributions store VM appliances in a variety of ways. 
The OpenStack project~\cite{openstack} provides appliance discovery, registration and delivery, 
via its ``Glance'' service. The appliances can be stored in simple filesystems or 
object-storage systems like ``Swift''\@. Both metadata about registered appliances 
and the appliances themselves are exposed via the Glance API\@.

Eucalyptus~\cite{eucalyptus}, an open-source IaaS cloud distribution, provides an Amazon S3 interface to its 
``Walrus'' storage service. Virtual machine images are stored/retrieved using HTTP put/get.

OpenNebula~\cite{opennebula}, another IaaS distribution, uses the concept of a Datastore for storing appliances. 
Multiple datastores can be created backed by one of a selection of supported filesystem types. 
The method used to store and retrieve appliances depends on the type of datastore used (e.g. filesystem, 
iSCSI, Ceph). A set of ``transfer manager'' scripts that handle this interaction are provided for 
each of the types.

In StratusLab appliances can be stored in any web accessible (via HTTP(S)) location. The location of the appliance 
is contained in the metadata published in the Marketplace\@. This concept makes it possible 
to share appliances between StratusLab users, and also between users of different cloud 
infrastructures, because of the open accessibility and portability of StratusLab appliances (Sec.~\ref{sec:portable-appliances}). 

\subsection{Appliance Factories (generation)}

New appliances can be created from scratch or by modifying existing appliances.
A number of different tools, like Bitnami~\cite{bitnami}, Puppet~\cite{puppet}, and the StratusLab client,
can be used for this.

Bitnami provides pre-built, ready to deploy appliances with components installed by default.
These can be used as a base appliance that a user can download and customize with their 
required software and configuration.

StratusLab provides ``image recipes'' for the automated creation of appliances. 
These recipes can be used to create minimal appliances for several
Linux distributions. StratusLab use the recipies to automatically update 
the set of base appliances provided. The StratusLab client also includes a tool that  
automates the creation of a new appliance from an existing base appliance.
The tool installs packages specified by the user and executes a user-defined  
script to configure the newly generated appliance. The new appliance is made available 
in the StratusLab storage service.

\subsection{Appliance Transport}

Appliances could be stored inside or outside of the cloud. In the latter case, 
appliances must be transported to the cloud for use.

The vmcaster/vmcatcher tools developed within the HEPIX Virtualisation Working
Group ~\cite{hepixbooktransfer}, use the concept of subscriptions to an 
appliance list. It makes the download and transport of an appliance 
from the appliance list similar to using a system package manager.
The downloaded appliances are
verified against their X509 signatures and cached.

In StratusLab, appliances are transported from a web
server or from cloud storage. Based on the appliance identifier in the Marketplace,
the transport of the appliance is done transparently by the cloud infrastructure.
The downloaded appliances are then verified and cached in the persistent disk storage,
ensuring that the transport of an appliance is done only once.

\subsection{Appliance Registry}

The StratusLab Marketplace is at the center of the appliance handling
mechanisms in the StratusLab cloud distribution. It contains metadata 
about appliances and serves as a registry for shared appliances. 
In order to use and/or share an appliance, its metadata must be registered 
in the Marketplace\@. 

Once an appliance is created, StratusLab provides simple tools for 
building, cryptographically signing with a valid
certificate, and uploading the metadata to the Marketplace\@.  The Marketplace
validates the metadata entry and verifies the 
email address of the endorser.  If all the checks pass, 
the metadata will then be visible in the Marketplace and other users 
can search for the entry.


\section{Marketplace Image Management}
\label{sec:approach}

{\em Start with what is different about the Marketplace: emphasis on
  sharing of images (especially in federated environments), adaptable
  to different domains (i.e. flexible metadata can be specialized,
  trust mechanisms are open to actors), and being neutral with respect
  to cloud implementations.  Describe very high level design of system
  and interactions with the actors and cloud services.  The technical
  note was organized around specific workflows; this may be good to
  keep for the paper.  The section should explicitly mention that
  ``third-party'' endorsers could be supported in this system.}

The Marketplace it at the center of the image handling mechanisms in
the StratusLab cloud distribution.  It contains metadata about images
and serves as a registry for shared images.  There are two primary use
cases for images in the cloud: 1) creating new images and making them
available and 2) instantiating a referenced image.

\subsection{Registering a New Image}

New machine or disk images can either be created from scratch
following the StratusLab guidelines or be modified versions of
existing images.  The workflow for creating an image and making it
available for other involves the following steps:

\paragraph{Image Creation} A new machine or disk image is created from
  scratch or from an existing image.  The \cmd{stratus-create-image}
  command can facilitate this process.

\paragraph{Transfer to Storage} The new image is transferred to network
  accessible storage.  This may be storage provided by a StratusLab
  cloud or storage provided by the user or users' institute.  The
  physical image must be available via one or more URLs\@.

\paragraph{Create Metadata} The metadata entry for the image must be
  created.  This is an XML file containing information about the image
  itself including the location URLs for the physical image.  The
  \cmd{stratus-build-manifest} command may be helpful here.

\paragraph{Sign Metadata} All of the metadata entries must be signed with a
  valid grid certificate.  The \cmd{stratus-sign-metadata} will sign
  the metadata file with a given certificate, inserting the
  certificate information automatically into the metadata file.

\paragraph{Upload Metadata} The signed metadata entry can then be uploaded
  to the StratusLab Marketplace\@.  This is done via an HTTP POST to a
  the Marketplace's \url{http://mp.example.org/metadata/} URL\@.

The Marketplace will then validate the entry and verify the email
address given in the image metadata.  Once all of the checks have been
done, the metadata will be visible in the Marketplace and other users
can search for the entry.

\subsection{Using a Registered Image}

Referencing an existing image and instantiating it involves the
following steps:

\paragraph{Request New Instance} StratusLab users will create new machine
  instances through the \cmd{stratus-run-instance} command.  Disk
  images can be used in the machine description to attach existing,
  public disks to a new instance.  In both cases, the user can supply
  an image identifier (e.g. ``MMZu9WvwKIro-rtBQfDk4PsKO7\_''), a
  Marketplace URL
  (e.g. \url{http://mp.example.org/metadata/?id=MMZu9WvwKIro-rtBQfDk4PsKO7\_},
  any URL that provides (directly or indirectly) a valid metadata
  entry (e.g. \url{http://tinyurl.com/my-favorite-image}), or the
  physical location of the image
  (e.g. \url{http://example.org/myimage}).

\paragraph{Resolve Entry} The URL given by the user must be resolved into
  one or more metadata entries associated with the image.  For any of
  the URLs that provide the metadata entry (or entries) directly, this
  is a simple HTTP GET on the URL\@.  For location based URLs, this
  will involve a search in the Marketplace either on the location URL
  or on a checksum calculated on the fly.

\paragraph{Apply Site Policy} The command \cmd{stratus-policy-image} will
  apply a configurable site policy to the image metadata.  If none of
  the metadata entries for a particular image pass the site policy,
  then the image will not be instantiated.

\paragraph{Download} Assuming that the image passed the site policy, then
  the \cmd{stratus-download-image} can be used to download a copy of
  the image based on the location URLs in the metadata entry.

Once the image is available, it will be instantiated and run.  All of
the work on the cloud side, it expected to happen in the clone hook of
OpenNebula\@.


\section{Design and Requirements}

{\em Dive into more detail about the design and various requirements
  for the system.}

The primary requirement of the Marketplace is that it permits all
users to search the range of available images for ones which satisfy
their requirements--avoiding the effort to create a customized image.
Equally important, the Marketplace provides a repository of image
metadata that can be used by cloud administrators to decide if a
particular image is trusted and can be run on their cloud.

Although the StratusLab cloud distribution must provide a mechanism
for storing and retrieving the image file themselves, a conscious
decision was made to separate the storage and transport of image file
from the Marketplace implementation.  Storing the image files outside
of the Marketplace:
\begin{itemize}
\item Makes it easier to scale the Marketplace implementation and to
  create mirrors of it.
\item Allows owners of the image to control access to the image
  itself.
\item Makes the Marketplace implementation independent of the
  transport protocol, allowing many different protocols to be
  supported.
\item Relieves the operator of the Marketplace from concerns related
  to image copyrights.
\end{itemize}
Related tools for downloading referenced images and validation of
those images must complement the Marketplace service.

\begin{table}
\caption{Requirements}
\label{tab:requirements}
\begin{tabular}{p{0.4\textwidth}}
\hline\hline
Allow anyone to upload valid metadata descriptions to the site.

\\ Valid descriptions must be signed by a grid certificate.  The
  endorser information must match the information in the certificate
  itself.  It may also be desirable to allow signatures with a PGP key
  or SSH key.

\\ Valid descriptions must contain a valid email address.  The
  service must confirm the email address for every upload of metadata.

\\ All valid descriptions must contain a creation date for the
  endorsement.  The server must only accept descriptions with a
  creation date more recent that the current latest.

\\ There may be several sets of metadata associated with a
  particular machine or disk image. This allows third parties to
  endorse images created by someone else. (E.g. VO endorses an image
  created by StratusLab.)

\\ Users can ``replace'' existing metadata descriptions by
  uploading a new signed description.  Nonetheless, all validated
  descriptions uploaded to the site must always be available to
  provide a history of the metadata evolution.

\\ Users must be able to search the metadata database on a
  reasonable subset of the possible keys.  Two required keys are the
  image identifier and the endorser's email address.

\\ Users must be able to download the original signed metadata in
  the RDF/XML format from the registry.  This is the only format that
  allows the metadata to be cryptographically verified.

\\ The registry should allow the metadata to be downloaded in
  alternate formats, notably JSON and HTML.

\\ Descriptions of available images should contain at least one
  location from which the image can be obtained.  Descriptions without
  a location are appropriate only if the image becomes unavailable.

\\ The service must be easy to use from all programming languages
  (including scripting languages) and usable from a web browser.

\\ The underlying schema for the metadata descriptions must be
  flexible and extensible.  The accounts for differing needs of users
  and eventual evolution of those needs.

\\

\hline\hline
\end{tabular}
\end{table}


\section{Implementation}
\label{sec:implementation}

To ensure that the service can be used easily with all programming
languages, the Marketplace exposes a RESTful~\cite{rest} interface
over standard HTTP(S).  This interface is implemented using
RESTlet~\cite{restlet}, a Java framework for RESTful services.

The user interface to the Marketplace, accessed through a Browser, is
implemented using a combination of FreeMarker~\cite{freemarker}, a
Java Template Engine Library, JQuery~\cite{jquery}, custom HTML, CSS
and JavaScript.

\subsection{Identifiers}

To ensure an unambiguous connection between the metadata entries
contained in the Marketplace and the described machine and disk
appliances, a unique identifier must be used that depends solely on
the appliance's contents.

For the purposes of the Marketplace and caching of appliances, an
identifier derived from the SHA-1 hash of the appliance file is used.  The
identifier is the 160-bit SHA-1 hash of the file (treating the
file as a binary file), extended with two zero bits to the left to
form a 162-bit value.  This 162-bit value is then encoded using the
base64url encoding to produce a 27 character identifier.  For
example, the SHA-1 hash (written in hex):
\begin{verbatim}
c319bbd5afc0a22ba3eaed0507c39383ec28eeff
\end{verbatim}
becomes the identifier:
\begin{verbatim}
MMZu9WvwKIro-rtBQfDk4PsKO7_
\end{verbatim}
using this algorithm.  Note that to avoid inadvertent or malicious
collisions of the SHA-1 additional checksum within the metadata
description should also be verified.

\subsection{Metadata}

Semantic web technologies were designed to manage metadata about
(third-party) resources identified with a URI.  Consequently, they are
ideally suited to this situation in which the Marketplace must manage
metadata about machine and disk appliances.  These technologies already
provide standard formats for the metadata (RDF/XML~\cite{rdfxml,
  rdfprimer, rdfschema}) and query languages (SeRQL~\cite{serql},
SPARQL~\cite{sparql}).  The Marketplace implementation makes use of the
OpenRdf Sesame~\cite{sesame} framework to provide search
capabilities over the metadata database.

To validate the metadata associated with a particular appliance, it is
necessary to cryptographically sign individual entries.  As the raw
format used for the metadata entries will be XML, the XML
Signature~\cite{xmlsig} specification is reused.  This is
conveniently a standard part of modern Java runtime environments.

Working with RDF also requires an agreed vocabulary to ensure a common
semantic meaning of the metadata tags.  The Dublin Core Metadata
Initiative has published a vocabulary~\cite{dcterms} that can be used
for much of the appliance metadata descriptions.  This is complemented by
a vocabulary specific to StratusLab.  Using RDF also allows additional
metadata fields to be specified (in separate namespaces) to complement
the standard fields.

\subsection{REST Resource URLs}

The mapping between URLs and service resources is the central part of
any RESTful service.  The resource mapping must provide convenient
access via a browser but also facilitate automated interactions with
tools.  Table~\ref{table:restmap} provides the URL mapping for the
Marketplace.  (The `delete' and `put' methods are not supported by any
URLs.)  Within the table ``identifier'' refers to the 27 character
image identifier, ``email'' refers to the endorser's email address,
and ``date'' refers to endorsement date written in the format {\tt
  yyyy-MM-ddThh:mm:ssZ}.  All of the URLs support an
XML and HTML formats.  In addition single metadata entries also 
provide a JSON representation.

\begin{table}
\caption{Core REST Resources}
\label{table:restmap}
\begin{center}
\begin{tabular}{ll}

\hline
\hline

\multicolumn{2}{l}{{\tt /}} \tnl
GET & redirects to /metadata resource \tnl
\hline

\multicolumn{2}{l}{{\tt /endorsers}} \tnl
GET & list of endorsers in database \tnl
OPTIONS & number of endorsers; last update time \tnl
\hline

\multicolumn{2}{l}{{\tt /endorsers/\replaceable{email}}} \tnl
GET & statistics about particular endorser \tnl
OPTIONS & number of entries; last update time \tnl
\hline

\multicolumn{2}{l}{{\tt /metadata/?\replaceable{query}}} \tnl
GET & list of identifiers and selected fields (query terms of \tnl
    & (identifer, email, and created can be used to refine list) \tnl
POST & create new metadata entry \tnl
OPTIONS & number of entries; last update time \tnl
\hline

\multicolumn{2}{l}{{\tt /metadata/\replaceable{identifier}/\replaceable{email}/\replaceable{date}}} \tnl
GET & unique metadata entry \tnl

\multicolumn{2}{l}{\tt /metadata/query} \tnl
GET & form for simple query of service \tnl
POST & submit query \tnl
\hline

\multicolumn{2}{l}{\tt /upload} \tnl
GET & form for browser upload of metadata entry \tnl
POST & create new entry via post to /metadata \tnl

\hline
\hline

\end{tabular}
\end{center}
\end{table}

\subsection{Storage and Query of Metadata}

Metadata uploaded to the Marketplace is first validated as described
above. Once accepted, the entry is written to the filesystem in a
temporary location as an XML file. A confirmation is then sent to the
email address contained in the endorsement field of the
metadata. Assuming the endorser completes the upload by visiting the
confirmation URL contained in the email, the metadata entry is moved
to the permanent data directory. A copy of the metadata, stripped of
the XML signature is then added to the Sesame repository. Stripping of
the signature is necessary as Sesame is unable to accept the signed
metadata. Storing the metadata to the repository allows simple queries
and queries with SPARQL~\cite{sparql} to be easily supported. A
request for a specific metadata entry in XML format results in the
original signed file stored on the filesystem being returned.


\section{Metadata Schema and Format}
\label{chap:metadata}

{\em Should this be a separate section or just be included in the
  implementation section?  In any case, it should cover the main
  points of the metadata, leaving open a possible
  transition/incorporation of OVF.  It should emphasize that the
  format is extensible to allow various domains to develop their own
  conventions.}

Sharing machine and disk images requires standardized, trusted
metadata to allow users to find appropriate images and to allow system
administrators to judge the suitability of them.

As stated in the previous chapter, the metadata descriptions are in
RDF/XML~\cite{rdfxml} format and cryptographically signed following
the XML Signature~\cite{xmlsig} specification.  The connection between
the described image and the metadata description is the image
identifier based on the SHA-1 hash.

Figure~\ref{fig:metadata-example} shows an (unsigned) example of the
metadata description.  The first element is the description of the
image containing information about the image file, contained operating
system, and location.  It also contains the endorsement of the
information with information on who endorsed the image and when.  The
email of the endorser is used as the key and is consequently a
required element of the description.  A digital signature element
(``xmldsig:Signature'') follows the ``rdf:Description'' element for
signed metadata entries.  (Relevant XML namespaces are listed in
Table~\ref{table:ns}.)

The entries in the Marketplace deal with {\em individual} images.  If
it is desired that collections of images are signed, then one
possibility is to include in each individual entry references to the
other image descriptions in the collection.  This allows the full
collection to be reconstructed from any individual entry.  One method
of doing this is shown in the example metadata description.

\begin{table}
\caption{XML Namespaces and Prefixes}
\label{table:ns}
\begin{center}
\begin{tabular}{ll}
\hline
  rdf & {\tt http://www.w3.org/1999/02/22-rdf-syntax-ns\#} \tnl
  dcterms & {\tt http://purl.org/dc/terms/} \tnl
  slreq & {\tt http://mp.stratuslab.eu/slreq\#} \tnl
  slterms & {\tt http://mp.stratuslab.eu/slterms\#} \tnl
\hline
\end{tabular}
\end{center}
\end{table}

\begin{figure}
\begin{center}
\tiny
\begin{verbatim}
<?xml version="1.0" encoding="UTF-8" standalone="no"?>
<rdf:RDF 
  xmlns:rdf="http://www.w3.org/1999/02/22-rdf-syntax-ns#"
  xmlns:dcterms="http://purl.org/dc/terms/"
  xmlns:slreq="http://mp.stratuslab.eu/slreq#"
  xmlns:slterms="http://mp.stratuslab.eu/slterms#"
  xmlns:ex="http://example.org/"
  xml:base="http://mp.stratuslab.eu/">

  <rdf:Description rdf:about="#MMZu9WvwKIro-rtBQfDk4PsKO7_">

    <dcterms:identifier>MMZu9WvwKIro-rtBQfDk4PsKO7_</dcterms:identifier>

    <slreq:bytes>100</slreq:bytes>

    <slreq:checksum rdf:parseType="Resource">
      <slreq:algorithm>SHA-1</slreq:algorithm>
      <slreq:value>c319bbd5afc0a22ba3eaed0507c39383ec28eeff</slreq:value>
    </slreq:checksum>

    <slreq:endorsement rdf:parseType="Resource">
      <dcterms:created>2011-01-24T09:59:42Z</dcterms:created>
      <slreq:endorser rdf:parseType="Resource">
        <slreq:email>jane.tester@example.org</slreq:email>
        <slreq:subject>CN=Jane Tester,OU=...</slreq:subject>
        <slreq:issuer>CN=Jane Tester,OU=...</slreq:issuer>
      </slreq:endorser>
    </slreq:endorsement>

    <dcterms:type>machine</dcterms:type>

    <dcterms:valid>2011-07-23T10:59:42Z</dcterms:valid>

    <dcterms:publisher>StratusLab</dcterms:publisher>
    <dcterms:title>linux-with-my-apps</dcterms:title>
    <dcterms:description>A 32-bit ttylinux...</dcterms:description>

    <slterms:location>http://example.org/...</slterms:location>

    <slterms:serial-number>0</slterms:serial-number>
    <slterms:version>1.0</slterms:version>

    <slterms:hypervisor>kvm</slterms:hypervisor>

    <slterms:inbound-port>443</slterms:inbound-port>
    <slterms:outbound-port>25</slterms:outbound-port>
    <slterms:icmp>8</slterms:icmp>

    <slterms:os>ttylinux</slterms:os>
    <slterms:os-version>9.7</slterms:os-version>
    <slterms:os-arch>i486</slterms:os-arch>

    <slterms:deprecated>security issue with app</slterms:deprecated>

    <ex:other-info>additional metadata</ex:other-info>
    <ex:yet-more>still more info</ex:yet-more>
    
    <ex:relatedImages rdf:parseType="Resource">
      <dcterms:identifier>MMZu9WvwKIro-rtBQfDk4PsKO7_</dcterms:identifier>
      <dcterms:identifier>NMZu9WvwKIro-rtBQfDk4PsKO7_</dcterms:identifier>
      <dcterms:identifier>OMZu9WvwKIro-rtBQfDk4PsKO7_</dcterms:identifier>
      <dcterms:identifier>PMZu9WvwKIro-rtBQfDk4PsKO7_</dcterms:identifier>
    </ex:relatedImages>

  </rdf:Description>
</rdf:RDF>
\end{verbatim}
\end{center}
\caption{Example (Abbreviated) Metadata Description}
\label{fig:metadata-example}
\end{figure}

\subsection{Signing and Validating StratusLab Metadata files}

For signing and validating metadata files we are using XML
Signature~\cite{xmlsig} specification.  Commands to support metadata
signatures have been written in Java as recent Java virtual machines
contain an API implementing this standard.

Metadata files can be signed using grid certificates (in PKCS12
format), PGP key pairs, or DSA/RSA key pairs.  Verification and
validation automatically detects signature algorithm and type of
private key used for signing metadata files, verifies the metadata
file and prints, for grid certificates, the DN of the user who signed
the metadata file.

\subsection{Metadata Elements}

Where possible the Dublin Core metadata vocabulary~\cite{dcterms} has
been used for the metadata description.  Table~\ref{table:dcterms}
shows the terms taken from the Dublin Core specification.  Additional
terms have been defined by StratusLab to complete the metadata
description (see Table~\ref{table:slterms}).

Additional terms can be added to the metadata descriptions, but they
should appear in their own XML namespaces.  {\em This allows for
  application-specific metadata and also evolution of the standard
  schema.}  These should appear after the endorsement element in the
description.

\begin{table}
\caption{Metadata Elements from Dublin Core}
\label{table:dcterms}
\begin{center}
\begin{tabular}{lllllp{4cm}}
\hline
NS & qname & freq. & XSD & Constraints & Notes \tnl
\hline
dcterms & identifier & 1 & string & valid identifier & image identifier \tnl
dcterms & isReplacedBy & ? & string & valid identifier & image identifier for replacement image \tnl
dcterms & replaces & ? & string & valid identifier & image identifier for image replaced by this one \tnl
dcterms & isVersionOf & ? & string & valid identifier & image identifier for parent image \tnl
dcterms & valid & ? & dateTime &  XML DateTime format & expiration date for image metadata \tnl
dcterms & title & ? & string &  & short title for humans \tnl
dcterms & description & 1 & string & & longer description of the image \tnl
dcterms & type & 1 & string & ``machine'' or ``disk'' & type of the described image \tnl
dcterms & creator & ? & string & & name of image or metadata record creator \tnl
dcterms & created & ? & string & & date when metadata record was created \tnl
dcterms & publisher & ? & string & & publisher (group, experiment, project) of image \tnl
dcterms & format & 1 & string & & format of machine or disk image \tnl
\hline
\end{tabular}
\end{center}
\end{table}

\begin{table}
\caption{Metadata Elements from StratusLab Schema}
\label{table:slterms}
\begin{center}
\begin{tabular}{lllllp{8cm}}
\hline
NS & qname & freq. & XSD & Constraints & Notes \tnl
\hline
slreq & endorsement & 1 & complex & & endorsement information \tnl
slreq & endorser & 1 & complex & & endorser information \tnl
slreq & bytes & 1 & positive integer & & bytes of described image \tnl
slreq & checksum & + & string & lowercase hex digits only & checksum in hex with algorithm prefix \tnl
slreq & email & 1 & string & & email address of the metadata record creator \tnl
slreq & subject & 1 & string & & certificate subject \tnl
slreq & issuer & + & string & & certificate issuer \tnl
\hline
slterms & location & * & URI & & location hint for download (none if unavailable) \tnl
slterms & serial-number & ? & non-negative integer & & numeric index of image within a series \tnl
slterms & version & ? & string & & version of the image \tnl
slterms & hypervisor & ? & string & & appropriate hypervisors for machine image \tnl
slterms & inbound-port & * & unsigned short & 0 for all & required inbound port \tnl
slterms & outbound-port & * & unsigned short & 0 for all & required outbound port \tnl
slterms & icmp & * & unsigned byte & & ICMP packet types \tnl
slterms & os-arch & ? & string & & OS architecture \tnl
slterms & os-version & ? & string & & OS version \tnl
slterms & os & ? & string & & OS \tnl
slterms & deprecated & ? & string & & reason that image is deprecated (missing if OK) \tnl
\hline
\end{tabular}
\end{center}
\end{table}


\section{Security Considerations}
\label{sec:security}

The Marketplace and the information contained within the Marketplace
play a key role in maintaining the security of the cloud
infrastructures.  However, the security policies both for the users
and for the cloud administrators can vary widely and {\em
  consequently, the Marketplace itself does not define or enforce any
  security policy.}  It instead provides the appliance metadata
allowing both users and cloud administrators to make informed
decisions about the appliances.

To maintain confidence in the information provided by the Marketplace,
it must securely provide complete, accurate information about the
appliances.  We have identified a number of security concerns and
describe how the Marketplace solves them. 

\subsection{Altered Appliances}

Because the appliance metadata is separated from the appliance
contents, there is a danger that the appliance contents could be
altered, either accidently or maliciously.  As described above,
the appliance identifier is based on the SHA-1 hash of the image
ensuring a very reliable link between the metadata and the
appliance contents.

Although modifying an image while maintaining the SHA-1 hash is
difficult, it is a remote possibility, allowing the altered appliance
to masquerade as the original.  To minimize this possibility,
additional information is provided in the metadata descriptions: the
size of the file in bytes and the MD5, SHA-1, SHA-256, and SHA-512
hash values.  The likelihood that someone can create an altered
appliance with exactly the same length and multiple checksums is
miniscule.

\subsection{Verification of Uploaded Metadata}

When new metadata entries are uploaded to the Marketplace, the service
validates the entry before accepting it.  To validate the entries, the server:
\begin{itemize}
\item Verifies that the metadata is a valid RDF/XML file, following
  the defined schemas and conventions,
\item Accepts only entries endorsed after the most recent entry for a
  particular appliance are accepted, and 
\item Confirms the email address of the entry.
\end{itemize}
Only validated metadata entries are visible from the standard
Marketplace interface.

\subsection{Altered Appliance Timeline}

The Marketplace must ensure that all of the data concerning an
appliance is available.  By design, the Marketplace never removes
metadata entries--the entire appliance timeline is always available.
By default however, only the current endorsements are provided as
these are the entries needed to make decisions about the validity of
an image.  

The Marketplace must ensure that the data transmitted to users is not
altered, for example by a third-party removing deprecation notices
from the returned information.  To prevent this, the Marketplace only
transmits information over a communication channel secured via TLS\@.

In addition, it does not allow the history of an appliance to be
altered.  As described above, the Marketplace does not accept entries
with a timestamp earlier than the latest entry in the timeline.  It
also validates the endorser to avoid one endorser from impersonating
another.

\subsection{Compromised Marketplace Server}

If someone were to take control of the Marketplace server, he could
not alter individual metadata entries as those are signed by the
endorsers' private keys are not stored on the server.  However, he
could delete entries making, for instance, deprecated images appear
valid.

This is a significant risk and the server must be operated according
to modern best practices to avoid this.  In addition, backups of the
information should be kept and periodically compared to the current
information to detect any such attack.


\section{Production Experience \& Planned Improvements}
\label{sec:production}

In parallel with its software development, the collaboration operates
a federated cloud infrastructure with sites in Orsay, France and
Athens, Greece\@.  These sites share a common user authentication
framework and Marketplace allowing users to allocate resources and to
use appliances on either site.  The collaboration uses this production
cloud to validate its software in real world conditions.

\begin{table}
\caption{}
\label{tab:statistics}
\begin{center}
\begin{tabular}{ll}
\hline
number of endorsers & 44 \\
number of current appliances & 105 \\
number of deprecated appliances & 170 \\
number of expired appliances & 832 \\
\hline
\end{tabular}
\end{center}
\end{table}

As a core service, the Marketplace is heavily used by both members of
the collaboration and users of the federated cloud.  Some statistics
on its use can be found in Table~\ref{tab:statistics}.  Overall the
service has performed well, with problems being addressed as the
software evolves over time.  Some outstanding points and potential
solutions are described below.

\subsection{Availability}

Having a central Marketplace instance allows users to easily find all
of the appliances from a single location.  Similarly, it allows image
creators to upload the metadata just once.  However, the Marketplace
is consulted everytime a new machine instance is launched to check if
an appliance has been deprecated.  Consequently if the Marketplace is
not available, new images cannot be started on any cloud relying on
the Marketplace\@.  Future iterations of the Marketplace must provide
redundancy and high-availability of the Marketplace service. 

To provide for this a replication scheme will be implemented that will 
allow multiple Marketplace instances to be run, each maintaining its own 
local copy of the global metadata database. As all the information required 
to rebuild the metadata database stored in Sesame is the set of raw metadata 
files, it is only these that need be replicated.  A potential solution to provide this
replication is to use a Git repository as the core `database' for the
metadata entries. This only requires that a Marketplace database be
updated periodically from a local clone of the global repository.

\subsection{Data Protection}

By design the appliance metadata is considered public.  In reality,
however, there are both users and administrators that would like to
restrict the visibility of the appliance metadata.  Many cloud
administrators would like to run a ``private'' Marketplace to limit
the visibility of certain appliances while still taking advantage of
the central, public Marketplace instance.  There must be a mechanism
for federating Marketplace instances in the future. 

This requirement could be integrated with the replication mechanism described above.  
For a public Marketplace, the repository could be hosted on a public service, 
such as GitHub~\cite{github} with each replicated server pulling/pushing 
from that repository.  For a private Marketplace, the repository can exist 
only on the local disk.

\subsection{Appliance Quality}

Although the metadata contains a significant amount of information
about an appliance, it does not contain information about how well the
appliance functions for users.  A common request has been to add
``social'' features to the Marketplace to allow users of an appliance
to leave comments and to signal problems with the appliance itself.  
A possible approach to add these features without overly complicating 
the Marketplace implementation would be to make use of an external service 
such as Disqus~\cite{disqus}.

\subsection{Stable URLs}

An important ability is to trace the evolution of an image through its
metadata. Currently modifying an appliance results in a new identifier
being created, which will then be used in the updated metadata. This
breaks the link with the previous version. The recently added `tag'
functionality goes someway to solving this.  The metadata entries for
a particular appliance can be found by retrieving all entries
containing a particular tag. This element is optional, however, and
may not be added to all entries. A more rigourous solution would be to
make use of the Dublin Core terms `replaces' and `isReplacedBy'. This
would provide a link in each metadata entry to the previous and next
entries in the lifetime of an appliance. Using these terms could add make it 
more difficult for a user to create a metadata entry, and so should be 
automated as much as possible.


\section{Integration with Cloud Implementations}
\label{sec:related-components}

{\em Describe what components are necessary to integrate the
  Marketplace with a IaaS cloud implementation. This should emphasize
  that it could be integrated into different cloud stacks.  This
  should be easy!}

To integrate the Marketplace with cloud deployments, three additional
components are necessary:

\paragraph{Resolver} This resolves a given URL into a set of metadata
  entries.  Currently only Marketplace metadata URLs are supported, so
  this is simply an HTTP GET on the URL and not a separate component.

\paragraph{Download} This is implemented in the
  \cmd{stratus-download-image} command.  It supports any transport
  supported natively by python.

\paragraph{Policy Engine} This is implemented in \cmd{stratus-policy-image}
  command.  It will validate the metadata entries against a site
  policy.  It currently implements signature validation, endorser
  white and black lists, and checksum blacklist.

All of these are used on the cloud side and are not visible to end
users.  Cloud administrators, however, may find them useful in certain
circumstances.


\section{Production Experiences}
\label{sec:production}

{\em This section should describe the experiences (good and bad) of
  running a Marketplace in production.  This should include the
  StratusLab experience, the EGI experience, and IBM/CiaB experience
  (if possible).}


% An example of a floating figure using the graphicx package.
% Note that \label must occur AFTER (or within) \caption.
% For figures, \caption should occur after the \includegraphics.
% Note that IEEEtran v1.7 and later has special internal code that
% is designed to preserve the operation of \label within \caption
% even when the captionsoff option is in effect. However, because
% of issues like this, it may be the safest practice to put all your
% \label just after \caption rather than within \caption{}.
%
% Reminder: the "draftcls" or "draftclsnofoot", not "draft", class
% option should be used if it is desired that the figures are to be
% displayed while in draft mode.
%
%\begin{figure}[!t]
%\centering
%\includegraphics[width=2.5in]{myfigure}
% where an .eps filename suffix will be assumed under latex, 
% and a .pdf suffix will be assumed for pdflatex; or what has been declared
% via \DeclareGraphicsExtensions.
%\caption{Simulation Results}
%\label{fig_sim}
%\end{figure}

% Note that IEEE typically puts floats only at the top, even when this
% results in a large percentage of a column being occupied by floats.


% An example of a double column floating figure using two subfigures.
% (The subfig.sty package must be loaded for this to work.)
% The subfigure \label commands are set within each subfloat command, the
% \label for the overall figure must come after \caption.
% \hfil must be used as a separator to get equal spacing.
% The subfigure.sty package works much the same way, except \subfigure is
% used instead of \subfloat.
%
%\begin{figure*}[!t]
%\centerline{\subfloat[Case I]\includegraphics[width=2.5in]{subfigcase1}%
%\label{fig_first_case}}
%\hfil
%\subfloat[Case II]{\includegraphics[width=2.5in]{subfigcase2}%
%\label{fig_second_case}}}
%\caption{Simulation results}
%\label{fig_sim}
%\end{figure*}
%
% Note that often IEEE papers with subfigures do not employ subfigure
% captions (using the optional argument to \subfloat), but instead will
% reference/describe all of them (a), (b), etc., within the main caption.


% An example of a floating table. Note that, for IEEE style tables, the 
% \caption command should come BEFORE the table. Table text will default to
% \footnotesize as IEEE normally uses this smaller font for tables.
% The \label must come after \caption as always.
%
%\begin{table}[!t]
%% increase table row spacing, adjust to taste
%\renewcommand{\arraystretch}{1.3}
% if using array.sty, it might be a good idea to tweak the value of
% \extrarowheight as needed to properly center the text within the cells
%\caption{An Example of a Table}
%\label{table_example}
%\centering
%% Some packages, such as MDW tools, offer better commands for making tables
%% than the plain LaTeX2e tabular which is used here.
%\begin{tabular}{|c||c|}
%\hline
%One & Two\\
%\hline
%Three & Four\\
%\hline
%\end{tabular}
%\end{table}


% Note that IEEE does not put floats in the very first column - or typically
% anywhere on the first page for that matter. Also, in-text middle ("here")
% positioning is not used. Most IEEE journals/conferences use top floats
% exclusively. Note that, LaTeX2e, unlike IEEE journals/conferences, places
% footnotes above bottom floats. This can be corrected via the \fnbelowfloat
% command of the stfloats package.

\section{Conclusions}
\label{sec:conclusions}

{\em Describe any conclusions that we want to draw from our
  development and operation of the service.  This should also include
  ``future work'' with things like federation of Marketplace servers,
  etc.}


% use section* for acknowledgement
\section*{Acknowledgements}

StratusLab was co-funded by the European Commission through the 7th
Framework Programme (Capacities), contract number INFSO-RI-261552 from
June 2010 to May 2012\@.  The authors also gratefully acknowledge
support from the other members of the StratusLab collaboration and
their institutes.




% trigger a \newpage just before the given reference
% number - used to balance the columns on the last page
% adjust value as needed - may need to be readjusted if
% the document is modified later
%\IEEEtriggeratref{8}
% The "triggered" command can be changed if desired:
%\IEEEtriggercmd{\enlargethispage{-5in}}

% references section

% can use a bibliography generated by BibTeX as a .bbl file
% BibTeX documentation can be easily obtained at:
% http://www.ctan.org/tex-archive/biblio/bibtex/contrib/doc/
% The IEEEtran BibTeX style support page is at:
% http://www.michaelshell.org/tex/ieeetran/bibtex/
\bibliographystyle{IEEEtran}
% argument is your BibTeX string definitions and bibliography database(s)
\bibliography{stratuslab-federated-images}
%
% <OR> manually copy in the resultant .bbl file
% set second argument of \begin to the number of references
% (used to reserve space for the reference number labels box)
%\begin{thebibliography}{1}
%
%\bibitem{IEEEhowto:kopka}
%H.~Kopka and P.~W. Daly, \emph{A Guide to \LaTeX}, 3rd~ed.\hskip 1em plus
%  0.5em minus 0.4em\relax Harlow, England: Addison-Wesley, 1999.
%
%\end{thebibliography}


% that's all folks
\end{document}
