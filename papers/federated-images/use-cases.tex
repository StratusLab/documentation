\section{Actors and Use Cases}
\label{sec:use-cases}

Three core use cases---publishing an appliance, using an appliance, and
authorizing an appliance---expose the required features and the actors
for appliance management in both federated and non-federated
environments.  The primary actors and their roles are:
\begin{description}
\item[Creator] Makes a new appliance and desires to publish the image
  for her own or someone else's use.
\item[Endorser] Validates an appliance against certain criteria and
  issues an endorsement to this effect.
\item[User] A scientist or engineer who wants to find and to use
  existing appliances.
\item[Admin] The cloud administrator responsible for
  maintaining the cloud services and the security of the platform.
\end{description}

\subsection{Publishing an Appliance}

Despite automation, appliance creation is a tedious, error-prone, and
lengthy process.  Creators with the expertise to create new
appliances often want to share their appliances with a larger
community for wider use and better testing.  Doing so requires
publishing the appliance.

To publish the appliance, the creators must put the appliance itself
in an accessible location and provide metadata about the appliance to a
central registry.  At a minimum, the metadata needs to include
information about the appliance contents, service configuration, and
access parameters.  The appliance contents and metadata may be 
provided either as separate files or a single file.

\subsection{Using an Appliance}

If possible, users want to avoid the effort required to create an
appliance by reusing an existing instance.  A central registry allows
users to discover appropriate appliances based, for example, on the
operating system used, what services are enabled, etc.  Appliance
metadata is crucial for finding appropriate appliances.

Users also want to verify the origin of the metadata and the integrity
of the appliance's image.  An endorsement of the image is critical for
establishing trust in the image itself.  Often the creator is also an
endorser of an image, but images can and often are endorsed by
multiple people.  This allows, for instance, third party certification
of appliances.

Once users find a suitable appliance, they want to run an instance of
that appliance on a cloud infrastructure.  This requires transport of
the appliance's image to the cloud infrastructure.  Although users
can do this manually, cloud infrastructures should handle
the transport transparently given the appliance identifier.

\subsection{Authorizing an Appliance}

Most users of cloud infrastructures have little or no experience with
system management.  They are unfamiliar with best practices and
techniques for securing machines, for example limiting SSH access and
configuration of firewalls.  Consequently, cloud administrators have a
strong interest in ensuring that users run appliances that have been
built with these best practices in mind.

Before allowing a user to start an appliance, cloud administrators
will want to authorize that particular appliance.  Based on appliance
metadata, administrators can define a policy that suits their needs,
ranging from very restrictive policies that allow only appliances endorsed by
one particular person, to open policies that permit all 
appliances except those with known security problems.
