\section{Actors and Use Cases}
\label{sec:usecases}

Appliance management, being a core feature of IaaS clouds, affects
nearly everyone within the IaaS cloud environment.  For the purpose of
discussing the motivating use cases of the Marketplace, we identify
the following actors:
\begin{description}
\item[Creator] Makes a new appliance and desires to publish the image
  for her own or someone else's use.
\item[Endorser] Validates an appliance against certain criteria and
  issues an endorsement to this effect.
\item[User] A scientist or engineer who wants to find and to use
  existing appliances.
\item[Admin.] The cloud administrator responsible for
  maintaining the cloud services and the security of the platform.
\end{description}
These actors play key roles in the three identified use cases for the
Marketplace: publishing an appliance, using an appliance, and
authorizing an appliance. 

\subsection{Publishing an Appliance}

Typically, creating a new appliance with the proper contextualization
and that is secure is tedious.  The process involves a large number of
steps and mistakes often require restarting the lengthy process.
Automation helps but image creation remains a time consuming process.

Creators with the expertise to create new appliances often want to
share their appliances with a larger community, in effect publishing
their appliance for use by others.  To publish the appliance, the
creators must put the appliance itself in an accessible location and
must provide metadata about it.  The metadata is uploaded to a central
registry to allow users to discover appliances. 

In addition to the metadata supplied by the image creator, users will
want to verify the origin of that metadata and that the image hasn't
been tampered with.  An endorsement of the image is critical for
establishing trust in the image itself.  Often the creator is also an
endorser of an image, but images can and often are endorsed by
multiple people.  This allows, for instance, third party certification
of appliances.

\subsection{Using an Appliance}

Users want to search the full list of available appliances to find one
that suits their needs.  This requires a central location (or possibly
several locations) where a comprehensive list of appliances is
available.  Once users find a suitable appliance, they want to run an
instance of that appliance on a cloud infrastructure, ideally by only
using the appliance's identifier.

To determine if an appliance is suitable, users will want to know what
operating system is used in the image, what services are enabled, how
to access the image, etc.  Appliance metadata is crucial for allowing
users to find appropriate appliances. 

\subsection{Authorizing an Appliance}

Most users of IaaS cloud infrastructures have little or no experience
with system management.  This means that they are unfamilar with best
practices and techniques for securing machines, for example limiting
SSH access and configuration of firewalls.  Because of this, cloud
administrators have a strong interest in ensuring that users run
appliances that have been built with these best practices in mind.

Before allowing a user to start an appliance, cloud administrators
will want to authorize that particular appliance.  Again, the
appliance metadata is crucial.  Based on this metadata,
administrator's can define a policy that suits their needs.  On one
side they can create very restrictive policies like only allowing
appliances endorsed by a particular person to be run.  On the other
they can define an open policy that only forbids appliances with known
security problems.
